\chapter{Errori a regime}

\section{Esercizi svolti}
\begin{esercizio}
Siano dati
\[
	G(s) = \frac{2}{s^2+2} \qquad H(s) = s+2 \qquad e(t) = u(t) -3y(t)
\]
Si determinino \(e_p, e_v, e_a\).

\paragraph{Soluzione}
Si osserva che la retroazione \emph{non è unitaria}. Questo significa che si ha un errore \(\gamma\) ed è specificato tra i dati come il coefficente di \(y(t)\) in \(e(t)\) (altrimenti sarebbe stato \(\gamma = H(0)\)): quindi \(\gamma = 3\).
A questo punto è possibile ricavare \(G_{eq}(s)\) per poter poi procedere con gli errori:
\begin{align*}
	G_{eq}(s) &= \frac{\gamma G(s)}{1+G(s)\bigl(H(s)-\gamma\bigr)} =
			\frac{\frac{6}{s^2+2}}{1+\frac{2}{s^2+2}\bigl(s+2-3\bigr)} =
			\frac{\frac{6}{s^2+2}}{\frac{2^2+2+2s-2}{s^2+2}} =
		  	\frac{6}{s(s+2)}
\end{align*}
e si osserva che il sistema è \emph{di tipo 1} perché ha un polo nell'origine.
Posso procedere con gli errori di posizione, velocità e accelerazione:
\begin{align*}
	k_p &= \lim_{s\to0} G_{eq}(s) = \lim_{s\to0} \frac{6}{s(s+2)} = +\infty \implies e_p = \frac{1}{k_p} = 0 \\
	e_v &= \lim_{s\to0} \frac{1}{s G_{eq}(s)} = \lim_{s\to0} \frac{s(s+2)}{6s} = \frac{1}{3} \\
	e_a &= \lim_{s\to0} \frac{s^2 G_{eq}(s)} = \lim_{s\to0} \frac{s(s+2)}{6s^2} = +\infty
\end{align*}
\[\implies \begin{cases}
	e_p = 0 \\
	e_v = \frac{1}{3} \\
	e_a = +\infty
\end{cases}\]
\end{esercizio}

\begin{esercizio}
Siano dati
\[
	G_c(s) = k \qquad G_p(s) = \frac{1}{s(s+4)} \qquad H(s) = 1
\]
Si determini quando il sistema è \emph{asintoticamente stabile}, il valore di \(k\) tale che \(e_v = 0.1\) e \(e_p\), \(e_a\).

\paragraph{Soluzione}
Per la stabilità ricorro a \(G_0(s)\):
\begin{align*}
	G_0(s) &= \frac{G(s)}{1+G(s)H(s)} \quad \text{con } G(s) = G_c(s)G_p(s) \\
	\implies G_0(s) &= \frac{\frac{k}{s(s+4)}}{1+\frac{k}{s(s+4)}} =
		\frac{\frac{k}{s(s+4)}}{\frac{s(s+4)+k}{s(s+4)}} =
		\frac{k}{s^2+4s+k}
\end{align*}
Noto che per \(k>0\) i coefficienti sono tutti positivi, quindi per il \emph{Lemma di Routh} il sistema è asintoticamente stabile per \(k>0\).

Dato che il sistema è a retroazione unitaria:
\begin{align*}
	G_{eq}(s) &= G(s) = \frac{k}{s(s+4)} \longrightarrow \text{sistema di \emph{tipo 1}} \\
	&\implies \begin{cases}
		e_p = 0 \\
		e_v = \frac{4}{k} = 0.1 \implies k=40 \\
		e_a = +\infty
	\end{cases}
\end{align*}
\end{esercizio}

\begin{esercizio}
Siano dati
\[
	G_c(s) = k \qquad G_p(s) = \frac{s+3}{(s+1)^2(s+p)} \qquad
	H(s) = s+2 \qquad e(t) = u(t) -3y(t)
\]

\paragraph{Verifica la stabilità del sistema.}
\[
	G_0(s) = \frac{G(s)}{1+G(s)H(s)} =
		\frac{\frac{k(s+3)}{(s+1)^2(s+p)}}{1+\frac{k(s+3)}{(s+1)^2(s+p)}(s+2)} =
		\frac{k(s+3)}{(s+1)^2(s+p)+k(s+3)(s+2)}
\]
Per il \emph{Lemma di Routh} il sistema è asintoticamente stabile per \(k>0,\,p>0\).

\paragraph{Stabilire la relazione tra \(p\) e \(k\) tale che \(e_p = 0.1\).}
\begin{align*}
	G_{eq}(s) &= \frac{\gamma G(s)}{1+G(s) \bigl(H(s) - \gamma\bigr)} \quad
			\text{con } \gamma = 3 \\
	G_{eq}(s) &= \frac{\frac{3k(s+3)}{(s+1)^3(s+p)}}{1+\frac{k(s+3)}{(s+1)^3(s+p)}(s+2-3)} = \\
		  &= \frac{3k(s+3)}{(s+1)^3(s+p)+k(s+3)(s-1)} \;
			\longrightarrow \text{sistema di \emph{tipo 0}} \\
	k_p &= \lim_{s\to0} G_{eq}(s) = \frac{9k}{p-3k} \\
	\implies e_p &= \frac{1}{1+k_p} = \frac{1}{1+\frac{9k}{p-3k}} = \frac{p-3k}{p-3k+9k} = \frac{p-3k}{p+6k} = \frac{1}{10} \\
	\implies p &= 4k
\end{align*}

\paragraph{Verificare se esiste una relazione tra \(p\) e \(k\) tale che \(e_v\) ed \(e_a\) siano finiti.}
\[
	G_{eq}(s) = \frac{3k(s+3)}{(s+1)^3(s+p)+k(s+3)(s-1)} = \frac{3k(s+3)}{s^3 +(p+2+k)s^2 +(1+2p+2k)s +p-3k}
\]
Per far sì che \(e_v\) sia finito, il sistema deve essere di tipo 1, quindi \(p-3k=0 \rightarrow p = 3k\). Per far sì che \(e_p\) sia finito, il sistema deve essere di tipo 2, quindi \(1+2p+2k=0\). Ponendo a sistema le equazioni:
\[\begin{cases}
	p = -\frac{3}{8} \\
	k = -\frac{1}{8}
\end{cases}\]
\(p<0\), \(k<0\), quindi la relazione non è valida.
\end{esercizio}

\begin{esercizio}
Siano dati
\[
	G_c(s) = 1 \qquad G_p(s) = \frac{10}{s(s+10)} \qquad
	H(s) = h > 0 \qquad e(t) = u(t) -hy(t)
\]
\paragraph{Determinare \(h\) tale che \(e_v\leq0.2\).}
Si nota che la retroazione è \emph{parametrica}, quindi si deve porre \(G_{eq}(s)=hG(s)\).
\begin{align*}
	G_{eq}(s) &= hG(s) = \frac{10h}{s(s+10)} \longrightarrow \text{sistema di \emph{tipo 1}} \\
	e_v &= \lim_{s\to0} \frac{1}{1+sG_{eq}(s)} =
		\lim_{s\to0} \frac{1}{1+s\frac{10h}{s(s+10)}} = \\
	    &= \lim_{s\to0} \frac{s+10}{s+10+10h} =
	    	\frac{1}{h} \leq \frac{1}{5}
	    	\rightarrow h \geq 5
\end{align*}

\paragraph{Con \(h=5\) stabilire il valore a regime, il valore massimo di sovraelongazione e il tempo di picco.}
.
\end{esercizio}
