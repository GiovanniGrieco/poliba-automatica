\chapter{Criterio di Routh}

Per un sistema è importante il fattore dell'\emph{accuratezza}, questo considerato
quando si considerano i poli del sistema e si analizzano il loro comportamento,
specie se il sistema fosse a \emph{retroazione} (o anello chiuso) e avesse possibili
disturbi/parametri che potrebbero variare variare.

Il \emph{luogo delle radici} è la descrizione dei valori dei poli che assumono al variare
dei parametri di sistema. Con la sua osservazione su un piano di Gauss, è possibile
determinare la stabilità del sistema e quindi comprendere il range possibile dei
parametri per un suo uso in sicurezza.

Per il tracciamento del luogo delle radici, si preferisce studiare il sistema
\emph{al limite della stabilità}, quindi è importante conoscere possibili poli
sull'asse immaginario/punto di origine e sul semipiano positivo. Poiché tali
punti corrispondono al limite della stabilità di un sistema in retroazione, per
determinarli si può far ricorso al \emph{criterio di Routh}.

È consigliato limitare il guadagno parametrico del sistema, questo perché si
possono verificare problemi di saturazione e i poli tendono verso il semipiano
destro (ovvero verso gli asintoti, quindi al limite della stabilità), quindi
l'intero sistema diventa instabile. L'obiettivo è comunque scegliere dei valori
parametrici tale da migliore il più possibile la precisione del sistema a regime.

Per applicare il criterio di Routh si usa la \emph{tabella di taratura} del luogo,
che consente di determinare il miglior guadagno parametrico e la corrispondente
stabilità del sistema.

\paragraph{Lemma di Routh}
Condizione necessaria affinché tutte le radici del polinomio \(q(s) = 0\) siano a
parte reale \(\Re s_i < 0\) è che i coefficienti \(a_i > 0\).

\begin{esempio}[Caso generale]
Sia dato il sistema generico
\[
	G(s) = \frac{b_n s^n + \dots + b_0}{a_n s^n + a_{n-1}s^{n-1} + \dots + a_1 s + a_0}
\]
Se si trovassero le radici del denominatore con \(a_i > 0\), il sistema è
sicuramente stabile.
La condizione è sufficiente per \(n=1\) e \(n=2\):
\begin{itemize}
	\item con \(a_1 s + a_0 = 0 \colon s = -\frac{a_0}{a_1}\)
	\item con \(a_2 s^2 + a_1 s + a_0 = 0 \colon s_{1,2} = -\frac{a_1 \pm \sqrt{a^2_1 -4a_0 a_2}}{2a_2}\)
\end{itemize}
Per sistemi più grandi si usa la \emph{tabella di Routh}:
\[\begin{array}{r|rrr}
	s^n 	& a_n 	  & a_{n-2} & \dots	\\
	s^{n-1} & a_{n-1} & a_{n-3} & \dots 	\\
	s^{n-2} & b_1 	  & b_2     & \dots 	\\
	\vdots 	& c_1
\end{array}\]
dove
\[
	b_1 = \frac{a_{n-1} a_{n-2} - a_n a_{n-3}}{a_{n-1}} \qquad
	c_1 = \frac{b_1 a_{n-3} - a_{n-1}b_2}{b_1}
\]

\end{esempio}

\begin{esempio}
Si considera il seguente sistema:
\[
	G(s) = \frac{s^5 + 3}{s^6 + 2s^5 + 2s^4 - s^3 - 2s - 2}
\]
Per verificare il criterio di Routh bisogna considerare la tabella di Routh e
seguire una specie di algoritmo:
\begin{itemize}
	\item Per la prima riga si considera la sequenza di coefficienti ad indice pari
	\item Per la seconda riga i coefficienti di indice dispari
	\item Per la cella \((2,1)\) si esegue \((2,1)\cdot(1,2) - (1,1)\cdot(2,2)\).
		È possibile dividere o moltiplicare per la cella \((2,1)\) o
		sottomultipli, a patto che \emph{il segno non vari}.
	\item Per le celle successive sulla stessa riga, si trasla l'operazione
		mantenendo costante la prima colonna.
	\item Per le righe successive si riutilizzano i punti sopra citati.%
		\footnote{La cella di \(s^0\) ha quasi sempre lo stesso valore
			dell'ultima cella presente nella riga di \(s^2\)}
\end{itemize}
\[\begin{array}{r|rrrr}
	\tikzmark{g22m6}s^6 &   1 &   2 &  0 & -2	\\
	\tikzmark{g22m5}s^5 &   2 & - 1 & -2 	\\
	\tikzmark{g22m4}s^4 &   5 &   2 & -4 	\\
	\tikzmark{g22m3}s^3 & - 9 & - 2		\\
	\tikzmark{g22m2}s^2 &   8 & -36		\\
	\tikzmark{g22m1}s^1 & -85			\\
	\tikzmark{g22m0}s^0 & -36
\end{array}\]
\begin{tikzpicture}[overlay, remember picture, yshift=.25\baselineskip, shorten >=.5pt, shorten <=.5pt]
	\draw [->] ({pic cs:g22m6}) [bend right] to node[left]{\scriptsize p} ({pic cs:g22m5});
	\draw [->] ({pic cs:g22m5}) [bend right] to node[left]{\scriptsize p} ({pic cs:g22m4});
	\draw [->] ({pic cs:g22m4}) [bend right] to node[left]{\scriptsize v} ({pic cs:g22m3});
	\draw [->] ({pic cs:g22m3}) [bend right] to node[left]{\scriptsize v} ({pic cs:g22m2});
	\draw [->] ({pic cs:g22m2}) [bend right] to node[left]{\scriptsize v} ({pic cs:g22m1});
	\draw [->] ({pic cs:g22m1}) [bend right] to node[left]{\scriptsize p} ({pic cs:g22m0});
\end{tikzpicture}
\begin{itemize}
	\item Si segnano le permanenze (p) e le variazioni (v) di segno tra la prima cella
		di una riga e la cella della seguente.
	\item Le permanenze indicano il numero di poli con \(\Re s < 0\), mentre
		le variazioni i poli con \(\Re s > 0\).
\end{itemize}
In questo caso si hanno 3 permanenze e 3 variazioni: il sistema è \emph{instabile}.
\end{esempio}

\begin{esempio} Sia dato il seguente sistema:
\[
	G(s) = \frac{k}{s^3 - 12s +16}
\]
Il parametro \(k\) presente al numeratore non deve interessare perché si stanno
considerando solo i poli.

Si nota che manca il coefficiente di \(s^2\), che è \(0\).
Si procede con la tabella di Routh:
\[\begin{array}{r|rr}
	s^3 & 1 & -12 \\
	s^2 & 0 & 16
\end{array}\]
A questo punto è necessario fermarsi perché lo \(0\) non ha segno. Per ovviare ciò,
si segue un metodo di sostituzione della riga:
\begin{itemize}
	\item Si moltiplica la riga per \((-1)^h\) con \(h\) il numero di zeri
		incontrati fin'ora. (\(0;\; -16\))
	\item Si traslano a sinistra le celle (quelle in testa vanno in coda): (\(-16;\; 0\))
	\item Si effettua la somma in colonna tra la riga originale e quella
		ottenuta con questo metodo:
		\[\begin{array}{rr|r}
			  0 & 16 & +\\
			-16 &  0 & =\\
			\midrule
			-16 & 16
		\end{array}\]
	\item La riga ottenuta sostituisce quella originale.
\end{itemize}
\[\begin{array}{r|rr}
	\tikzmark{g23m3} s^3 &   1 & -12 \\
	\tikzmark{g23m2a}s^2 &   0 &  16 \\
	\midrule
	\tikzmark{g23m2b}s^2 & -16 &  16 \\
	\tikzmark{g23m1} s^1 & -11 	  \\
	\tikzmark{g23m0} s^0 &  16
\end{array}\]
\begin{tikzpicture}[overlay, remember picture, yshift=.25\baselineskip, shorten >=.5pt, shorten <=.5pt]
	\draw [->] ({pic cs:g23m3})  [bend right] to node[left]{\scriptsize v} ({pic cs:g23m2b});
	\draw [->] ({pic cs:g23m2b}) [bend right] to node[left]{\scriptsize p} ({pic cs:g23m1});
	\draw [->] ({pic cs:g23m1})  [bend right] to node[left]{\scriptsize v} ({pic cs:g23m0});
\end{tikzpicture}
Sono presenti due variazioni e una permanenza: il sistema è \emph{instabile}.
\end{esempio}

\begin{esempio} Si applichi il criterio di Routh al seguente sistema:
\[
	G(s) = \frac{s^5 + 4s + 1}{s^6 + 2s^5 + 8s^4 + 12s^3 + 20s^2 + 16s + 16}
\]
\[\begin{array}{r|rrrr}
	s^6 & 1 &  8 & 20 & 16	\\
	s^5 & 2 & 12 & 16 	\\
	s^4 & 1 &  6 &  8	\\
	s^3 & 0 &  0
\end{array}\]
A questo punto è necessario fermarsi perché si ha una riga completamente nulla.
Questo significa che esistono delle radici puramente immaginarie simmetriche
rispetto l'origine.

Si considerano i coefficienti della riga precedente e si costruisce un'equazione
di variabile \(s\): \(s^4 + 6s^2 + 8 = 0\). Derivandola si ha \(4s^3 + 12s = 0\).
La coppia di coefficienti di questa equazione costituiscono la nuova riga che
sostituisce quella nulla. Si nota che i coefficienti hanno multiplo in comune 4,
quindi dividento tutto per 4 si ha \((1;\; 3)\).
\[\begin{array}{r|rrrr}
	\tikzmark{g24m6} s^6 & 1 &  8 & 20 & 16	\\
	\tikzmark{g24m5} s^5 & 2 & 12 & 16 	\\
	\tikzmark{g24m4} s^4 & 1 &  6 &  8	\\
			 s^3 & 0 &  0		\\
	\midrule
	\tikzmark{g24m3} s^3 & 1 &  3		\\
	\tikzmark{g24m2} s^2 & 3 &  8		\\
	\tikzmark{g24m1} s^1 & 1			\\
	\tikzmark{g24m0} s^0 & 8
\end{array}\]
\begin{tikzpicture}[overlay, remember picture, yshift=.25\baselineskip, shorten >=.5pt, shorten <=.5pt]
	\draw [->] ({pic cs:g24m6}) [bend right] to node[left]{\scriptsize p} ({pic cs:g24m5});
	\draw [->] ({pic cs:g24m5}) [bend right] to node[left]{\scriptsize p} ({pic cs:g24m4});
	\draw [->] ({pic cs:g24m4}) [bend right] to node[left]{\scriptsize p} ({pic cs:g24m3});
	\draw [->] ({pic cs:g24m3}) [bend right] to node[left]{\scriptsize p} ({pic cs:g24m2});
	\draw [->] ({pic cs:g24m2}) [bend right] to node[left]{\scriptsize p} ({pic cs:g24m1});
	\draw [->] ({pic cs:g24m1}) [bend right] to node[left]{\scriptsize p} ({pic cs:g24m0});
\end{tikzpicture}
Nonostante siano tutte permanenze, bisogna tenere in considerazione che si è
sostituita una riga vuota e, come detto sopra, si hanno dei poli sull'asse
immaginario. Per verificarlo, basta considerare le soluzioni dell'equazione
precedente \(4s^3 + 12s = 0\) che ha soluzioni \(s_1 = 0\) e \(s_{2,3} = \pm\jmath\sqrt{3}\).
Per questo motivo il sistema è \emph{semplicemente stabile}.
\end{esempio}

\begin{esempio} Si applichi il criterio di Routh al seguente sistema:
\[
	G(s) = \frac{s^5 + 4s + 1}{s^5 + 5s^4 + 4s^3 + 120s^2 + 3s + 315}
\]
\[\begin{array}{r|rrr}
	\tikzmark{g25m5} s^5 & 1 & 4 & 3 \\
	\tikzmark{g25m4} s^4 & \cancelto{1}{5} & \cancelto{24}{120} & \cancelto{63}{315} \\
	\tikzmark{g25m3} s^3 & \cancelto{-1}{-20} & \cancelto{-3}{-60} \\
	\tikzmark{g25m2} s^2 & \cancelto{1}{21} & \cancelto{3}{63} \\
			 s^1 & 0 \\
	\midrule
	\tikzmark{g25m1} s^1 & 2 \\
	\tikzmark{g25m0} s^0 & 3
\end{array}\]
\begin{tikzpicture}[overlay, remember picture, yshift=.25\baselineskip, shorten >=.5pt, shorten <=.5pt]
	\draw [->] ({pic cs:g25m5}) [bend right] to node[left]{\scriptsize p} ({pic cs:g25m4});
	\draw [->] ({pic cs:g25m4}) [bend right] to node[left]{\scriptsize v} ({pic cs:g25m3});
	\draw [->] ({pic cs:g25m3}) [bend right] to node[left]{\scriptsize v} ({pic cs:g25m2});
	\draw [->] ({pic cs:g25m2}) [bend right] to node[left]{\scriptsize p} ({pic cs:g25m1});
	\draw [->] ({pic cs:g25m1}) [bend right] to node[left]{\scriptsize p} ({pic cs:g25m0});
\end{tikzpicture}
Il sistema presenta 3 permanenze e 2 variazioni, quindi è \emph{instabile}.
\end{esempio}

\section{Esercizi svolti}
\exercise{} Si applichi il criterio di Routh a
\[
	P(s) = s^4 + 4s^3 + 3s^2 + 8s + 5
\]
\[\begin{array}{r|rrr}
	\tikzmark{e21m4} s^4 & 1 & 3 & 5 \\
	\tikzmark{e21m3} s^3 & 4 & 8 \\
	\tikzmark{e21m2} s^2 & \cancelto{1}{4} & \cancelto{5}{20} \\
	\tikzmark{e21m1} s^1 & -12 \\
	\tikzmark{e21m0} s^0 & 5
\end{array}\]
\begin{tikzpicture}[overlay, remember picture, yshift=.25\baselineskip, shorten >=.5pt, shorten <=.5pt]
	\draw [->] ({pic cs:e21m4}) [bend right] to node[left]{\scriptsize p} ({pic cs:e21m3});
	\draw [->] ({pic cs:e21m3}) [bend right] to node[left]{\scriptsize p} ({pic cs:e21m2});
	\draw [->] ({pic cs:e21m2}) [bend right] to node[left]{\scriptsize v} ({pic cs:e21m1});
	\draw [->] ({pic cs:e21m1}) [bend right] to node[left]{\scriptsize v} ({pic cs:e21m0});
\end{tikzpicture}
Il sistema è \emph{instabile} con 2 poli aventi \(\Re s > 0\).


\exercise{} Si applichi il criterio di Routh a
\[
	P(s) = 8s^4 + 3s^3 + 7s^2 + 2s + 1
\]
\[\begin{array}{r|rrr}
	\tikzmark{e22m4} s^4 & 8 & 7 & 1 \\
	\tikzmark{e22m3} s^3 & 3 & 2 	 \\
	\tikzmark{e22m2} s^2 & 5 & 3	 \\
	\tikzmark{e22m1} s^1 & 1	 \\
	\tikzmark{e22m0} s^0 & 3
\end{array}\]
\begin{tikzpicture}[overlay, remember picture, yshift=.25\baselineskip, shorten >=.5pt, shorten <=.5pt]
	\draw [->] ({pic cs:e22m4}) [bend right] to node[left]{\scriptsize p} ({pic cs:e22m3});
	\draw [->] ({pic cs:e22m3}) [bend right] to node[left]{\scriptsize p} ({pic cs:e22m2});
	\draw [->] ({pic cs:e22m2}) [bend right] to node[left]{\scriptsize p} ({pic cs:e22m1});
	\draw [->] ({pic cs:e22m1}) [bend right] to node[left]{\scriptsize p} ({pic cs:e22m0});
\end{tikzpicture}
Il sistema presenta solo poli con \(\Re s < 0\), quindi è \emph{asintoticamente stabile}.


\exercise{} Si applichi il criterio di Routh a
\[
	P(s) = s^4 + s^3 + 5s^2 + 5s + 2
\]
\[\begin{array}{r|rrr}
	\tikzmark{e23m4} s^4 &  1 & 5 & 2 \\
	\tikzmark{e23m3} s^3 &  1 & 5	 \\
	s^2 &  0 & 2	 		 \\
	\midrule
	\tikzmark{e23m2} s^2 & -2 & 2	 \\
	\tikzmark{e23m1} s^1 & \cancelto{6}{12} \\
	\tikzmark{e23m0} s^0 & 2
\end{array}\]
\begin{tikzpicture}[overlay, remember picture, yshift=.25\baselineskip, shorten >=.5pt, shorten <=.5pt]
	\draw [->] ({pic cs:e23m4}) [bend right] to node[left]{\scriptsize p} ({pic cs:e23m3});
	\draw [->] ({pic cs:e23m3}) [bend right] to node[left]{\scriptsize v} ({pic cs:e23m2});
	\draw [->] ({pic cs:e23m2}) [bend right] to node[left]{\scriptsize v} ({pic cs:e23m1});
	\draw [->] ({pic cs:e23m1}) [bend right] to node[left]{\scriptsize p} ({pic cs:e23m0});
\end{tikzpicture}
Il sistema presenta 2 poli con \(\Re s > 0\), quindi è \emph{instabile}.


\exercise{} Si applichi il criterio di Routh a
\[
	P(s) = s^5 + s^4 + 4s^3 + 4s^2 + 7s + 7
\]
\[\begin{array}{r|rrr}
	\tikzmark{e24m5} s^5 & 1 & 4 & 7 \\
	\tikzmark{e24m4} s^4 & 1 & 4 & 7 \\
			 s^3 & 0 & 0	 \\
	\midrule
	\tikzmark{e24m3} s^3 & \cancelto{1}{4} & \cancelto{2}{8} \\
	\tikzmark{e24m2} s^2 & 2 & 7	\\
	\tikzmark{e24m1} s^1 & -3	\\
	\tikzmark{e24m0} s^0 & 7
\end{array}\]
\begin{tikzpicture}[overlay, remember picture, yshift=.25\baselineskip, shorten >=.5pt, shorten <=.5pt]
	\draw [->] ({pic cs:e24m5}) [bend right] to node[left]{\scriptsize p} ({pic cs:e24m4});
	\draw [->] ({pic cs:e24m4}) [bend right] to node[left]{\scriptsize p} ({pic cs:e24m3});
	\draw [->] ({pic cs:e24m3}) [bend right] to node[left]{\scriptsize p} ({pic cs:e24m2});
	\draw [->] ({pic cs:e24m2}) [bend right] to node[left]{\scriptsize v} ({pic cs:e24m1});
	\draw [->] ({pic cs:e24m1}) [bend right] to node[left]{\scriptsize v} ({pic cs:e24m0});
\end{tikzpicture}
Il sistema presenta 2 poli con \(\Re s > 0\), quindi è \emph{instabile}.


\exercise{} Si applichi il criterio di Routh a
\[
	P(s) = s^5 + 2s^4 + 4s^3 + 4s^2 + 3s + 2
\]
\[\begin{array}{r|rrr}
	\tikzmark{e25m5} s^5 & 1 & 4 & 3 \\
	\tikzmark{e25m4} s^4 & 2 & 4 & 2 \\
	\tikzmark{e25m3} s^3 & \cancelto{1}{4} & \cancelto{1}{4} \\
	\tikzmark{e25m2} s^2 & \cancelto{1}{2} & \cancelto{1}{2} \\
			 s^1 & 0 	 \\
	\midrule
	\tikzmark{e25m1} s^1 & 2 	 \\
	\tikzmark{e25m0} s^0 & 1
\end{array}\]
\begin{tikzpicture}[overlay, remember picture, yshift=.25\baselineskip, shorten >=.5pt, shorten <=.5pt]
	\draw [->] ({pic cs:e25m5}) [bend right] to node[left]{\scriptsize p} ({pic cs:e25m4});
	\draw [->] ({pic cs:e25m4}) [bend right] to node[left]{\scriptsize p} ({pic cs:e25m3});
	\draw [->] ({pic cs:e25m3}) [bend right] to node[left]{\scriptsize p} ({pic cs:e25m2});
	\draw [->] ({pic cs:e25m2}) [bend right] to node[left]{\scriptsize p} ({pic cs:e25m1});
	\draw [->] ({pic cs:e25m1}) [bend right] to node[left]{\scriptsize p} ({pic cs:e25m0});
\end{tikzpicture}
Il sistema presenta, al limite della stabilità, due poli puramente immaginari dati
dall'equazione \(s^2 + 1 = 0\), questa utilizzata per sostituire la riga nulla.
Quindi il sistema è \emph{semplicemente stabile}.


\exercise{} Si applichi il criterio di Routh e si stabiliscano i poli per
\[
	P(s) = s^6 + s^5 + 3s^4 + 3s^3 + 3s^2 + 2s + 1
\]
\[\begin{array}{r|rrrr}
	\tikzmark{e26m6} s^6 &  1 & 3 & 3 & 1	\\
	\tikzmark{e26m5} s^5 &  1 & 3 & 2	\\
			 s^4 &  0 & 1 & 1	\\
	\midrule
	\tikzmark{e26m4} s^4 & -1 & 0 & -1	\\
	\tikzmark{e26m3} s^3 & \cancelto{1}{3} & \cancelto{1}{3} \\
	\tikzmark{e26m2} s^2 &  1 & 1		\\
			 s^1 &  0		\\
	\midrule
	\tikzmark{e26m1} s^1 &  2		\\
	\tikzmark{e26m0} s^0 &  1
\end{array}\]
\begin{tikzpicture}[overlay, remember picture, yshift=.25\baselineskip, shorten >=.5pt, shorten <=.5pt]
	\draw [->] ({pic cs:e26m6}) [bend right] to node[left]{\scriptsize p} ({pic cs:e26m5});
	\draw [->] ({pic cs:e26m5}) [bend right] to node[left]{\scriptsize v} ({pic cs:e26m4});
	\draw [->] ({pic cs:e26m4}) [bend right] to node[left]{\scriptsize v} ({pic cs:e26m3});
	\draw [->] ({pic cs:e26m3}) [bend right] to node[left]{\scriptsize p} ({pic cs:e26m2});
	\draw [->] ({pic cs:e26m2}) [bend right] to node[left]{\scriptsize p} ({pic cs:e26m1});
	\draw [->] ({pic cs:e26m1}) [bend right] to node[left]{\scriptsize p} ({pic cs:e26m0});
\end{tikzpicture}
Il sistema deve avere 6 poli (equazione di grado 6): infatti presenta 2 poli con
\(\Re s_i > 0\), 2 poli puramente immaginari \(s_{1,2} = \pm\jmath\) soluzioni
dell'equazione \(s^2 + 1 = 0\), ricavata per sostituire la riga nulla \(s^1\),
e 2 poli con \(\Re s_i < 0\). Per concludere, il sistema è \emph{instabile}.


\exercise{Parametrico}
Si applichi il criterio di Routh a
\[
	P(s) = s^3 + 5s^2 + 6s + k
\]
\[\begin{array}{r|rr}
	\tikzmark{e27m3} s^3 & 1 & 6 \\
	\tikzmark{e27m2} s^2 & 5 & k \\
	\tikzmark{e27m1} s^1 & 30-k  \\
	\tikzmark{e27m0} s^0 & k
\end{array}\]
\begin{tikzpicture}[overlay, remember picture, yshift=.25\baselineskip, shorten >=.5pt, shorten <=.5pt]
	\draw [->] ({pic cs:e27m3}) [bend right] to node[left]{\scriptsize p} ({pic cs:e27m2});
	\draw [->] ({pic cs:e27m2}) [bend right] to node[left]{\scriptsize ?} ({pic cs:e27m1});
	\draw [->] ({pic cs:e27m1}) [bend right] to node[left]{\scriptsize ?} ({pic cs:e27m0});
\end{tikzpicture}

\begin{itemize}
	\item Se \(k > 30\):
		\[
			\begin{cases}
				2 \text{ poli con } \Re s_i > 0 \\
				1 \text{ polo con } \Re s   < 0
			\end{cases} \implies \text{sistema \emph{instabile}}
		\]
	\item Se \(k = 30\):
		\[\begin{array}{r|rr}
			\tikzmark{e27bm3} s^3 & 1 & 6	\\
			\tikzmark{e27bm2} s^2 & 5 & 30	\\
					  s^1 & 0 	\\
			\midrule
			\tikzmark{e27bm1} s^1 & 10	\\
			\tikzmark{e27bm0} s^0 & 30
		\end{array}\]
		\begin{tikzpicture}[overlay, remember picture, yshift=.25\baselineskip, shorten >=.5pt, shorten <=.5pt]
			\draw [->] ({pic cs:e27bm3}) [bend right] to node[left]{\scriptsize p} ({pic cs:e27bm2});
			\draw [->] ({pic cs:e27bm2}) [bend right] to node[left]{\scriptsize p} ({pic cs:e27bm1});
			\draw [->] ({pic cs:e27bm1}) [bend right] to node[left]{\scriptsize p} ({pic cs:e27bm0});
		\end{tikzpicture}
		Le soluzioni dell'equazione \(5s^2 + 30 = 0\) sono puramente
		immaginari, quindi il sistema è \emph{semplicemente stabile}
	\item Se \(0 < k < 30\): 3 poli con \(\Re s_i < 0 \implies\) sistema \emph{asintoticamente stabile}
	\item Se \(k < 0\):
		\[
			\begin{cases}
				1 \text{ polo con } \Re s   > 0 \\
				2 \text{ poli con } \Re s_i < 0
			\end{cases} \implies \text{sistema \emph{instabile}}
		\]
	\item Se \(k = 0\):
		\[\begin{array}{r|rr}
			\tikzmark{e27cm3} s^3 & 1 & 6	\\
			\tikzmark{e27cm2} s^2 & 5 & 0	\\
			\tikzmark{e27cm1} s^1 & 30	\\
					  s^0 & 0	\\
			\midrule
			\tikzmark{e27cm0} s^0 & 30
		\end{array}\]
		\begin{tikzpicture}[overlay, remember picture, yshift=.25\baselineskip, shorten >=.5pt, shorten <=.5pt]
			\draw [->] ({pic cs:e27cm3}) [bend right] to node[left]{\scriptsize p} ({pic cs:e27cm2});
			\draw [->] ({pic cs:e27cm2}) [bend right] to node[left]{\scriptsize p} ({pic cs:e27cm1});
			\draw [->] ({pic cs:e27cm1}) [bend right] to node[left]{\scriptsize p} ({pic cs:e27cm0});
		\end{tikzpicture}
		La soluzione dell'equazione \(30s = 0\) coincide con l'origine,
		quindi il sistema è \emph{semplicemente stabile}
\end{itemize}


\exercise{Parametrico} Si applichi il criterio di Routh e si stabiliscano i poli per
\[
	G_0 (s) = \frac{s-1}{s^2 + 5s^2 + (k-6)s + k}
\]
\[\begin{array}{r|rr}
	\tikzmark{e28am3} s^3 & 	  1 & k-6 \\
	\tikzmark{e28am2} s^2 & 	  5 &   k \\
	\tikzmark{e28am1} s^1 & 2k-15 	  	  \\
	\tikzmark{e28am0} s^0 & k
\end{array}\]
\begin{tikzpicture}[overlay, remember picture, yshift=.25\baselineskip, shorten >=.5pt, shorten <=.5pt]
	\draw [->] ({pic cs:e28am3}) [bend right] to node[left]{\scriptsize p} ({pic cs:e28am2});
	\draw [->] ({pic cs:e28am2}) [bend right] to node[left]{\scriptsize ?} ({pic cs:e28am1});
	\draw [->] ({pic cs:e28am1}) [bend right] to node[left]{\scriptsize ?} ({pic cs:e28am0});
\end{tikzpicture}
\begin{itemize}
	\item Se \(k < 0\):
		\[\begin{cases}
			1 \text{ polo con } \Re s   > 0 \\
			2 \text{ poli con } \Re s_i < 0
		\end{cases} \implies \text{sistema \emph{instabile}}\]
	\item Se \(k = 0\):
		\[\begin{array}{r|rr}
			\tikzmark{e28bm3} s^3 &   1 & -6\\
			\tikzmark{e28bm2} s^2 &   5 & 0	\\
			\tikzmark{e28bm1} s^1 & -30	\\
					  s^0 &   0	\\
			\midrule
			\tikzmark{e28bm0} s^0 & -30
		\end{array}\]
		\begin{tikzpicture}[overlay, remember picture, yshift=.25\baselineskip, shorten >=.5pt, shorten <=.5pt]
			\draw [->] ({pic cs:e28bm3}) [bend right] to node[left]{\scriptsize p} ({pic cs:e28bm2});
			\draw [->] ({pic cs:e28bm2}) [bend right] to node[left]{\scriptsize v} ({pic cs:e28bm1});
			\draw [->] ({pic cs:e28bm1}) [bend right] to node[left]{\scriptsize p} ({pic cs:e28bm0});
		\end{tikzpicture}
		\[\begin{cases}
			1 \text{ polo con } \Re s > 0	\\
			1 \text{ polo all'origine} 	\\
			1 \text{ polo con } \Re s < 0
		\end{cases} \implies \text{sistema \emph{instabile}}\]
	\item Se \(0 < k < \frac{15}{2}\):
		\[\begin{cases}
			2 \text{ poli con } \Re s_i > 0 \\
			1 \text{ polo con } \Re s   < 0
		\end{cases} \implies \text{sistema \emph{instabile}}\]
	\item Se \(k = \frac{15}{2}\):
		\[\begin{array}{r|rr}
			\tikzmark{e28dm3} s^3 &  1 & \frac{15}{2} - 6	\\
			\tikzmark{e28dm2} s^2 &  5 & \frac{15}{2}	\\
					  s^1 &  0			\\
			\midrule
			\tikzmark{e28dm1} s^1 &  2			\\
			\tikzmark{e28dm0} s^0 & 15
		\end{array}\]
		\begin{tikzpicture}[overlay, remember picture, yshift=.25\baselineskip, shorten >=.5pt, shorten <=.5pt]
			\draw [->] ({pic cs:e28dm3}) [bend right] to node[left]{\scriptsize p} ({pic cs:e28dm2});
			\draw [->] ({pic cs:e28dm2}) [bend right] to node[left]{\scriptsize p} ({pic cs:e28dm1});
			\draw [->] ({pic cs:e28dm1}) [bend right] to node[left]{\scriptsize p} ({pic cs:e28dm0});
		\end{tikzpicture}
		\[\begin{cases}
			2 \text{ poli puramente immaginari} \\
			1 \text{ polo con } \Re s < 0
		\end{cases} \implies \text{sistema \emph{semplicemente stabile}}\]
	\item Se \(k > \frac{15}{2}\):
		\[\begin{cases}
			3 \text{ poli con } \Re s_i < 0
		\end{cases} \implies \text{sistema \emph{asintoticamente stabile}}\]
\end{itemize}


\exercise{Sistema a retroazione negativa} Dato il sistema
\begin{center}\begin{tikzpicture}[auto,node distance=2cm,>=latex']
	\node [input] (x) {};
	\node [sum, right of=x] (sum) {};
	\node [block, right of=sum, node distance=1.5cm] (G1) {\(G_c (s)\)};
	\node [block, right of=G1] (G2) {\(G_p (s)\)};
	\node [output, right of=G2] (y) {};
	\node [tmp, below of=G1, node distance=1cm] (tmp) {};
	\draw [->] (x) -- (sum);
	\draw [->] (sum) -- (G1);
	\draw [->] (G1) -- (G2);
	\draw [->] (G2) -- node[name=retro]{} (y);
	\draw [->] (retro) |- (tmp) -| node[pos=0.9,anchor=east]{\(-\)} (sum);
\end{tikzpicture}\end{center}
con
\begin{align*}
	G_c (s) &= k \frac{s+1}{s-2} \\
	G_p (s) &= \frac{1}{s^2 + 2s + 10}
\end{align*}
si applichi il criterio di Routh e si discutano i poli al variare del parametro \(k\).

\paragraph{Soluzione}
\(G_c\) e \(G_p\) sono in serie, quindi
\[
	G(s) = G_c(s)G_p(s) = \frac{k(s+1)}{(s-2)(s^2 + 2s + 10)}
\]
Si considera \((s-2)(s^2+2s+10)+k(s+1) = s^3 \cancel{-2s^2} \cancel{+2s^2} -4s +10s -20 +ks +k = s^3 +(k+6)s -20+k\),
quindi \[P(s) = s^3 +(k+6)s -20+k\]
\[\begin{array}{r|rr}
	s^3 & 1 & k+6 \\
	s^2 & 0 & -20+k \\
	\midrule
	s^2 & 20-k & -20+k \\
	s^1 & -k^2+13k+140
\end{array}\]
È possibile semplificare l'equazione \(-k^2 +13k +140 = 0\) trovando le sue radici:
\begin{align*}
	k^2 -13k -140 &= 0 \\
	k_{1,2} = \frac{13 \pm27}{2} &= \begin{cases} -7 \\ 20 \end{cases} \\
	\implies k^2 -13k -140 &= (k +7) (k -20) \\
	-k^2 +13k +140 &= (k +7) (20 -k) \\
	\implies \frac{-k^2 +13k +140}{20 -k} &= k+7
\end{align*}
Di conseguenza
\[\begin{array}{r|rr}
	s^3 &    1 &   k+6 \\
	s^2 & 20-k & -20+k \\
	s^1 &  k+7 	   \\
	s^0 & -20+k
\end{array}\]
\begin{itemize}
	\item Se \(k < -7\):
		\[\begin{cases}
			1 \text{ polo con } \Re s > 0
		\end{cases}\]
	\item Se \(k = -7\):
		\[\begin{array}{r|rr}
			\tikzmark{e29am3} s^3 &  1 & -1 \\
			\tikzmark{e29am2} s^2 &  1 & -1 \\
					  s^1 &  0	\\
			\midrule
			\tikzmark{e29am1} s^1 &  2	\\
			\tikzmark{e29am0} s^0 & -1
		\end{array}\]
		\begin{tikzpicture}[overlay, remember picture, yshift=.25\baselineskip, shorten >=.5pt, shorten <=.5pt]
			\draw [->] ({pic cs:e29am3}) [bend right] to node[left]{\scriptsize p} ({pic cs:e29am2});
			\draw [->] ({pic cs:e29am2}) [bend right] to node[left]{\scriptsize p} ({pic cs:e29am1});
			\draw [->] ({pic cs:e29am1}) [bend right] to node[left]{\scriptsize v} ({pic cs:e29am0});
		\end{tikzpicture}
		\[\begin{cases}
			2 \text{ poli con } \Re s > 0	\\
		\end{cases} \implies \text{sistema \emph{instabile}}\]
	\item Se \(-7 < k < 20\):
		\[\begin{cases}
			1 \text{ polo con } \Re s   > 0
		\end{cases} \implies \text{sistema \emph{instabile}}\]
	\item Se \(k = 20\):
		\[\begin{array}{r|rr}
			\tikzmark{e29bm3} s^3 &  1 & 26 \\
					  s^2 &  0 &  0 \\
			\midrule
			\tikzmark{e29bm2} s^2 &  3 & 26 \\
			\tikzmark{e29bm1} s^1 & 52	\\
			\tikzmark{e29bm0} s^0 & 26
		\end{array}\]
		\begin{tikzpicture}[overlay, remember picture, yshift=.25\baselineskip, shorten >=.5pt, shorten <=.5pt]
			\draw [->] ({pic cs:e29bm3}) [bend right] to node[left]{\scriptsize p} ({pic cs:e29bm2});
			\draw [->] ({pic cs:e29bm2}) [bend right] to node[left]{\scriptsize p} ({pic cs:e29bm1});
			\draw [->] ({pic cs:e29bm1}) [bend right] to node[left]{\scriptsize p} ({pic cs:e29bm0});
		\end{tikzpicture}
		\[\begin{cases}
			1 \text{ polo all'origine} 		\\
			2 \text{ poli puramente immaginari}
		\end{cases} \implies \text{sistema \emph{semplicemente stabile}}\]
	\item Se \(k > 20\):
		\[\begin{cases}
			2 \text{ poli con } \Re s_i > 0
		\end{cases}\]
	\item Il sistema non è asintoticamente stabile \(\forall k\)
\end{itemize}

\exercise{Parametrico} Si applichi il criterio di Routh e si discutino i poli al variare di \(k\) per
\[
	G(s) = \frac{250k (s-3)}{(s+3)(s^2 +50s +2500)}
\]
\begin{align*}
	P(s) &= 250k (s-3) + (s+3)(s^2 +50s +2500) = 		   \\
	     &= 250ks -705k +s^3 +3s^2 +50s^2 +150s +2500s +7500 = \\
	     &= s^3 +53s^2 +(250k +2650)s +7500 -750k
\end{align*}
\[\begin{array}{r|rr}
	\tikzmark{e210am3} s^3 & 1 & 250k +2650 \\
	\tikzmark{e210am2} s^2 & 53 & 7500 -750k \\
	\tikzmark{e210am1} s^1 & 280k +2659 \\
	\tikzmark{e210am0} s^0 & 10-k
\end{array}\]
\begin{tikzpicture}[overlay, remember picture, yshift=.25\baselineskip, shorten >=.5pt, shorten <=.5pt]
	\draw [->] ({pic cs:e210am3}) [bend right] to node[left]{\scriptsize p} ({pic cs:e210am2});
	\draw [->] ({pic cs:e210am2}) [bend right] to node[left]{\scriptsize ?} ({pic cs:e210am1});
	\draw [->] ({pic cs:e210am1}) [bend right] to node[left]{\scriptsize ?} ({pic cs:e210am0});
\end{tikzpicture}
\begin{itemize}
	\item Se \(k < -\frac{2659}{280}\):
		\[\begin{cases}
			2 \text{ poli con } \Re s_i > 0 \\
			1 \text{ polo con } \Re s   < 0
		\end{cases} \implies \text{sistema \emph{instabile}}\]
	\item Se \(k = -\frac{2659}{280}\):
		\[\begin{array}{r|rr}
			\tikzmark{e210bm3} s^3 & 1 & 250k + 2650 \\
			\tikzmark{e210bm2} s^2 & 53 & 7500 -750k \\
					   s^1 & 0 \\
			\midrule
			\tikzmark{e210bm1} s^1 & 106 \\
			\tikzmark{e210bm0} s^0 & 7500 -750k
		\end{array}\]
		\begin{tikzpicture}[overlay, remember picture, yshift=.25\baselineskip, shorten >=.5pt, shorten <=.5pt]
			\draw [->] ({pic cs:e210bm3}) [bend right] to node[left]{\scriptsize p} ({pic cs:e210bm2});
			\draw [->] ({pic cs:e210bm2}) [bend right] to node[left]{\scriptsize p} ({pic cs:e210bm1});
			\draw [->] ({pic cs:e210bm1}) [bend right] to node[left]{\scriptsize p} ({pic cs:e210bm0});
		\end{tikzpicture}
		\[\begin{cases}
			2 \text{ poli puramente immaginari} \\
			1 \text{ polo con } \Re s < 0
		\end{cases} \implies \text{sistema \emph{semplicemente stabile}}\]
	\item Se \(-\frac{2659}{280} < k < 10\):
		\[\begin{cases}
			3 \text{ poli con } \Re s_i < 0
		\end{cases} \implies \text{sistema \emph{asintoticamente stabile}}\]
	\item Se \(k = 10\):
		\[\begin{array}{r|rr}
			\tikzmark{e210cm3} s^3 & 1 & 5150 \\
			\tikzmark{e210cm2} s^2 & 53 & 0 \\
			\tikzmark{e210cm1} s^1 & 5459 \\
					   s^0 & 0 \\
			\midrule
			\tikzmark{e210cm0} s^0 & 5459
		\end{array}\]
		\begin{tikzpicture}[overlay, remember picture, yshift=.25\baselineskip, shorten >=.5pt, shorten <=.5pt]
			\draw [->] ({pic cs:e210cm3}) [bend right] to node[left]{\scriptsize p} ({pic cs:e210cm2});
			\draw [->] ({pic cs:e210cm2}) [bend right] to node[left]{\scriptsize p} ({pic cs:e210cm1});
			\draw [->] ({pic cs:e210cm1}) [bend right] to node[left]{\scriptsize p} ({pic cs:e210cm0});
		\end{tikzpicture}
		\[\begin{cases}
			1 \text{ polo all'origine} \\
			2 \text{ poli con } \Re s_i < 0
		\end{cases} \implies \text{sistema \emph{semplicemente stabile}}\]
	\item Se \(k > 10\):
		\[\begin{cases}
			1 \text{ polo con } \Re s   > 0 \\
			2 \text{ poli con } \Re s_i < 0
		\end{cases} \implies \text{sistema \emph{instabile}}\]
\end{itemize}


\exercise{Parametrico} Si applichi il criterio di Routh e si discutino i
poli al variare di \(k\) al seguente sistema
\[
	G_c(s) = k \qquad G_p(s) = \frac{s-1}{s (s+1) (s^2 +8s +25)}
\]

La funzione di trasferimento e \(P(s)\) sono dati da
\begin{align*}
	G(s) &= G_c(s) G_p(s) = \frac{k(s-1)}{s (s+1) (s^2 +8s +25)} \\
	P(s) &= k(s-1) + s (s+1) (s^2 +8s +25) = ks -k +(s^2 +s)(s^2 +8s +25) = \\
	     &= ks -k +s^4 +s^3 +8s^3 +8s^2 +25s^2 +25s = \\
	     &= s^4 +9s^3 +33s^2 +(25+k)s -k
\end{align*}
\[\begin{array}{r|rrr}
	s^4 & 1 & 33 & -k \\
	s^3 & 9 & 25+k \\
	s^2 & 272-k & -9k
\end{array}\]
Per ricavare la riga successiva:
\begin{align*}
	(272-k)(25+k) + 81k &= -k^2 +328k +6800 \\
	\implies k^2 -328k -6800 &= 0\\
	k_{1,2} = 164 \pm 36\sqrt{26} &\approx \begin{cases} 347.6 \\ -19.6 \end{cases}
\end{align*}
quindi
\[\begin{array}{r|rrr}
	\tikzmark{e211am4} s^4 & 1 & 33 & -k 	\\
	\tikzmark{e211am3} s^3 & 9 & 25+k 		\\
	\tikzmark{e211am2} s^2 & 272-k & -9k	\\
	\tikzmark{e211am1} s^1 & -k^2 +328k +6800	\\
	\tikzmark{e211am0} s^0 & -9k
\end{array}\]
\begin{tikzpicture}[overlay, remember picture, yshift=.25\baselineskip, shorten >=.5pt, shorten <=.5pt]
	\draw [->] ({pic cs:e211am4}) [bend right] to node[left]{\scriptsize p} ({pic cs:e211am3});
	\draw [->] ({pic cs:e211am3}) [bend right] to node[left]{\scriptsize ?} ({pic cs:e211am2});
	\draw [->] ({pic cs:e211am2}) [bend right] to node[left]{\scriptsize ?} ({pic cs:e211am1});
	\draw [->] ({pic cs:e211am1}) [bend right] to node[left]{\scriptsize ?} ({pic cs:e211am0});
\end{tikzpicture}
I casi sono
\[\begin{cases}
	272-k > 0 \rightarrow k < 272 \\
	-9k > 0 \rightarrow k < 0 \\
	-k^2 +328k +6800 > 0 \rightarrow 164 -36\sqrt{26} < k < 164 +36\sqrt{26}
\end{cases}\]

\begin{itemize}
	\item Se \(164-36\sqrt{26} < k < 0\): sistema \emph{asintoticamente stabile}
	\item Se \(k = 0\): 1 polo nell'origine \(\implies\) sistema \emph{semplicemente stabile}
	\item Se \(k = 164-36\sqrt{26}\): 2 poli puramente immaginari \(\implies\) sistema \emph{semplicemente stabile}
	\item Se \(0 < k < 272\): 1 polo con \(\Re s > 0 \implies\) sistema \emph{instabile}
	\item Se \(k = 272\): 1 polo nell'origine, 2 poli puramente immaginari e 1 polo con \(\Re s > 0\): sistema \emph{instabile}
	\item Se \(272 < k < 164+36\sqrt{26}\): 3 poli con \(\Re s_i > 0 \implies\) sistema \emph{instabile}
	\item Se \(k = 164+36\sqrt{26}\): 1 polo con \(\Re s > 0 \implies\) sistema \emph{instabile}
	\item Se \(k > 164+36\sqrt{26}\): 1 polo con \(\Re s > 0 \implies\) sistema \emph{instabile}
\end{itemize}


\exercise{} Si applichi il criterio di Routh a
\[
	G_0(s) = \frac{s-2}{s^5 +5s^4 +11s^3 +23s^2 +28s +12}
\]
\[\begin{array}{r|rrr}
	\tikzmark{e212m5} s^5 & 1 & 11 & 28 \\
	\tikzmark{e212m4} s^4 & 5 & 23 & 12 \\
	\tikzmark{e212m3} s^3 & \cancelto{1}{32} & \cancelto{4}{128} \\
	\tikzmark{e212m2} s^2 & 3 & 12	   \\
			 s^1 & 0	   \\
	\midrule
	\tikzmark{e212m1} s^1 & 6	   \\
	\tikzmark{e212m0} s^0 & 12
\end{array}\]
\begin{tikzpicture}[overlay, remember picture, yshift=.25\baselineskip, shorten >=.5pt, shorten <=.5pt]
	\draw [->] ({pic cs:e212m5}) [bend right] to node[left]{\scriptsize p} ({pic cs:e212m4});
	\draw [->] ({pic cs:e212m4}) [bend right] to node[left]{\scriptsize p} ({pic cs:e212m3});
	\draw [->] ({pic cs:e212m3}) [bend right] to node[left]{\scriptsize p} ({pic cs:e212m2});
	\draw [->] ({pic cs:e212m2}) [bend right] to node[left]{\scriptsize p} ({pic cs:e212m1});
	\draw [->] ({pic cs:e212m1}) [bend right] to node[left]{\scriptsize p} ({pic cs:e212m0});
\end{tikzpicture}
\[\begin{cases}
	2 \text{ poli puramente immaginari} \\
	3 \text{ poli con } \Re s_i < 0
\end{cases} \implies \text{sistema \emph{semplicemente stabile}}\]

