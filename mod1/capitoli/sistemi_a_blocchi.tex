\chapter{Diagrammi a blocchi}

I diagrammi a blocchi servono per descrivere in modo astratto i sistemi automatici.
È possibile descrivere all'interno del blocco una propria \emph{funzione di trasferimento},
e questa sarà preceduta da delle \emph{frecce entranti} che faranno da \emph{input} e
da delle \emph{frecce uscenti} che faranno da \emph{output}.

Ogni input e output, secondo la logica del sistema, può essere sommato o sottrato con
altri fattori, come altri input e/o output oppure disturbi.
Il seguente esempio mostra un sistema statico con solo un input e un output
(altresì definito \emph{sistema SISO}).
\begin{center}\begin{tikzpicture}[auto, node distance=2cm,>=latex']
	\node [input, name=rinput] (rinput) {};
	\node [block, right of=rinput] (controller) {\(G(s)\)};
	\node [output, right of=controller, node distance=2cm] (output) {};
	\draw [->] (rinput) -- node{\(X(s)\)} (controller);
	\draw [->] (controller) -- node [name=y] {\(Y(s)\)}(output);
\end{tikzpicture}\end{center}

In caso di strutture più complesse e con l'interazione di più grandezze si parla
di \emph{sistema interconnesso}.

Inoltre esistono diverse relazioni che regolano la struttura dei diagrammi a blocchi.

\section{Strutture equivalenti}
\subsection{Blocchi in parallelo}
\begin{center}\begin{tikzpicture}[auto, node distance=2cm,>=latex']
	\node [input, name=xinput] (xinput) {};
	\node [block, right of=xinput] (ctrl1) {\(G_1(s)\)};
	\node [block, below of=ctrl1, node distance=2cm] (ctrl2) {\(G_2(s)\)};
	\node [sum, right of=ctrl1, node distance=2cm] (sum) {};
	\node [output, right of=sum] (output) {};
	\draw [->] (xinput) -- node[name=z]{} node{\(X(s)\)} (ctrl1);
	\draw [->] (ctrl1) -- node[name=out1] {\(Y_1(s)\)} (sum);
	\draw [->] (z) |- (ctrl2);
	\draw [->] (ctrl2) -| node[pos=0.25,anchor=south]{\(Y_2(s)\)} (sum);
	\draw [->] (sum) -- node[] {\(Y(s)\)} (output);
\end{tikzpicture}\end{center}

Le equazioni espresse nello schema sono
\begin{align*}
	Y_1(s) &= G_1(s)\,X(s) \\
	Y_2(s) &= G_2(s)\,X(s) \\
	Y(s) &= Y_1(s) + Y_2(s) \\
	\implies Y(s) &= \sum_{i=1}^\infty G_i(s)\,X(s)
\end{align*}
e quindi si può semplificare lo schema con un unico blocco che ne faccia la somma:
\begin{center}\begin{tikzpicture}[auto, node distance=2cm,>=latex']
	\node [input, name=x] (x) {};
	\node [block, right of=x] (ctrl) {\(\sum_{i=0}^\infty G_i(s)\)};
	\node [output, right of=ctrl] (y) {};
	\draw [->] (x) -- node[]{\(X(s)\)} (ctrl);
	\draw [->] (ctrl) -- node[]{\(Y(s)\)} (y);
\end{tikzpicture}\end{center}

Si conclude che \emph{il parallelo di due o più blocchi è equivalente alla somma
algebrica delle f.d.t. (guadagni) dei rispettivi blocchi interessati}.

\subsection{Blocchi in serie (o in cascata)}
\begin{center}\begin{tikzpicture}[auto, node distance=2cm,>=latex']
	\node [input, name=x] (x) {};
	\node [block, right of=x] (ctrl1) {\(G_1(s)\)};
	\node [block, right of=ctrl1, node distance=3cm] (ctrl2) {\(G_2(s)\)};
	\node [output, right of=ctrl2] (y) {};
	\draw [->] (x) -- node[]{\(X(s)\)} (ctrl1);
	\draw [->] (ctrl1) -- node[]{\(Z(s)\)} (ctrl2);
	\draw [->] (ctrl2) -- node[]{\(Y(s)\)} (y);
\end{tikzpicture}\end{center}

Le equazioni espresse nello schema sono
\begin{align*}
	Y(s) &= G_2(s)Z(s) \\
	Z(s) &= G_1(s)X(s) \\
	\implies Y(s) &= \prod_{i=1}^\infty G_i(s)X(s)
\end{align*}
e quindi si può semplificare lo schema con un blocco che ne faccia il prodotto:
\begin{center}\begin{tikzpicture}[auto, node distance=2cm,>=latex']
	\node [input] (x) {};
	\node [block, right of=x] (ctrl) {\(\prod_{i=0}^\infty G_i(s)\)};
	\node [output, right of=ctrl] (y) {};
	\draw [->] (x) -- node[]{\(X(s)\)} (ctrl);
	\draw [->] (ctrl) -- node[]{\(Y(s)\)} (y);
\end{tikzpicture}\end{center}

Si conclude che \emph{una serie di blocchi equivale al prodotto algebrico delle
f.d.t. dei rispettivi blocchi interessati}.

\subsection{Scambio di nodi sommatori}
\begin{center}\begin{tikzpicture}[auto, node distance=2cm,>=latex']
	\node [input] (x) {};
	\node [sum, right of=x] (sum1) {};
	\node [input, above of=sum1] (y) {};
	\node [sum, right of=sum1] (sum2) {};
	\node [input, above of=sum2] (w) {};
	\node [input, right of=sum2] (z) {};
	\draw [->] (x) -- node[]{\(X(s)\)} (sum1);
	\draw [->] (y) -- node[]{\(Y(s)\)} (sum1);
	\draw [->] (sum1) -- (sum2);
	\draw [->] (w) -- node[]{\(W(s)\)} (sum2);
	\draw [->] (sum2) -- node[]{\(Z(s)\)} (z);
\end{tikzpicture}\end{center}

L'equazione è \(Z(s) = W(s) + \bigl(Y(s) + X(s)\bigr)\) che, per la proprietà associativa
dell'addizione, può essere rivalutata come
\begin{align*}
	Z(s) &= \bigl(X(s) + W(s)\bigr) + Y(s) \\
	Z(s) &= \bigl(Y(s) + W(s)\bigr) + X(s) \\
	Z(s) &= X(s) + Y(s) + W(s)
\end{align*}
\begin{center}\begin{tikzpicture}[auto,node distance=2cm,>=latex']
	\begin{scope}
		\node [input] (x) {};
		\node [sum, right of=x] (sum1) {};
		\node [input, above of=sum1] (w) {};
		\node [sum, right of=sum1] (sum2) {};
		\node [input, below of=sum2] (y) {};
		\node [output, right of=sum2] (z) {};
		\draw [->] (x) -- node[]{\(X(s)\)} (sum1);
		\draw [->] (w) -- node[]{\(W(s)\)} (sum1);
		\draw [->] (sum1) -- (sum2);
		\draw [->] (y) -- node[]{\(Y(s)\)} (sum2);
		\draw [->] (sum2) -- node[]{\(Z(s)\)} (z);
	\end{scope}
	\begin{scope}[xshift=5cm]
		\node [input] (y) {};
		\node [sum, right of=y, node distance=2cm] (sum1) {};
		\node [input, below of=sum1] (w) {};
		\node [sum, right of=sum1] (sum2) {};
		\node [input, above of=sum2] (x) {};
		\node [output, right of=sum2] (z) {};
		\draw [->] (y) -- node[]{\(Y(s)\)} (sum1);
		\draw [->] (w) -- node[]{\(W(s)\)} (sum1);
		\draw [->] (sum1) -- (sum2);
		\draw [->] (x) -- node[]{\(X(s)\)} (sum2);
		\draw [->] (sum2) -- node[]{\(Z(s)\)} (z);
	\end{scope}
	\begin{scope}[xshift=11cm]
		\node [input] (x) {};
		\node [sum, right of=x, node distance=2cm] (sum) {};
		\node [input, above of=sum] (y) {};
		\node [input, below of=sum] (w) {};
		\node [output, right of=sum] (z) {};
		\draw [->] (x) -- node[]{\(X(s)\)} (sum);
		\draw [->] (y) -- node[]{\(Y(s)\)} (sum);
		\draw [->] (w) -- node[]{\(W(s)\)} (sum);
		\draw [->] (sum) -- node[]{\(Z(s)\)} (z);
	\end{scope}
\end{tikzpicture}\end{center}

Si conclude che \emph{l'ordine dei blocchi collegati ad un sommatore è ininfluente
e possono essere collegati da un unico blocco sommatore}.

\subsection{Spostamento di un punto di prelievo rispetto a un blocco}
\subsubsection{A monte di un blocco}
\begin{center}\begin{tikzpicture}[auto,node distance=2cm,>=latex']
	\node [input] (x) {};
	\node [block, right of=x] (ctrl) {\(G(s)\)};
	\node [output, right of=ctrl] (y) {};
	\node [output, below of=x, node distance=1cm] (yy) {};
	\draw [->] (x) -- node[]{\(X(s)\)} (ctrl);
	\draw [->] (ctrl) -- node[name=out]{\(Y(s)\)} (y);
	\draw [->] (out) |- node[pos=0.8]{\(Y(s)\)} (yy);
\end{tikzpicture}\end{center}

Questo schema, dato che ha per entrambe le uscite equazione \(Y(s) = G(s)X(s)\)
può essere sostituito dall'equivalente
\begin{center}\begin{tikzpicture}[auto,node distance=2cm,>=latex']
	\node [input] (x) {};
	\node [block, right of=x, node distance=4cm] (ctrl1) {\(G(s)\)};
	\node [output, right of=ctrl1] (y) {\(Y(s)\)};
	\draw [->] (x) -- node[name=align,pos=0.5]{\(X(s)\)} (ctrl1);
	\draw [] (x) -- node[name=in,pos=0.85]{} (ctrl1);
	\draw [->] (ctrl1) -- node[]{\(Y(s)\)} (y);
	% secondo livello
	\node [block, below of=align] (ctrl2) {\(G(s)\)};
	\node [output, left of=ctrl2, node distance=1.8cm] (yy) {};
	\draw [->] (in) |- (ctrl2);
	\draw [->] (ctrl2) -- node[]{\(Y(s)\)} (yy);
\end{tikzpicture}\end{center}

\subsubsection{A valle di un blocco}
\begin{center}\begin{tikzpicture}[auto,node distance=2cm,>=latex']
	\node [input] (x) {};
	\node [block, right of=x, node distance=3cm] (ctrl) {\(G(s)\)};
	\node [output, right of=ctrl] (y) {};
	\draw [->] (x) -- node[name=in,pos=0.5]{\(X(s)\)} (ctrl);
	\draw [->] (ctrl) -- node[]{\(Y(s)\)} (y);
	% secondo livello
	\node [output, below of=x, node distance=1cm] (xx) {};
	\draw [->] (in) |- node[pos=0.7]{\(X(s)\)} (xx);
\end{tikzpicture}\end{center}

Questo schema, dato che ha per una uscita equazione \(Y(s) = G(s)X(s)\) mentre
per l'altra l'identità \(X(s) = X(s)\), può essere riscritta ponendo alla prima
equazione \(X(s) = \frac{1}{G(s)}Y(s)\). Di conseguenza lo schema equivalente è
\begin{center}\begin{tikzpicture}[auto,node distance=2cm,>=latex']
	\node [input] (x) {};
	\node [block, right of=x] (ctrl1) {\(G(s)\)};
	\node [output, right of=ctrl1] (y) {};
	\draw [->] (x) -- node[]{\(X(s)\)} (ctrl1);
	\draw [->] (ctrl1) -- node[name=out]{\(Y(s)\)} (y);
	% secondo livello
	\node [block, below of=ctrl1, node distance=1.5cm] (ctrl2) {\(\frac{1}{G(s)}\)};
	\node [output, left of=ctrl2] (xx) {};
	\draw [->] (out) |- (ctrl2);
	\draw [->] (ctrl2) -- node[]{\(X(s)\)} (xx);
\end{tikzpicture}\end{center}

\subsection{Spostamento di un nodo sommatore rispetto a un blocco}
\subsubsection{A monte di un blocco}
\begin{center}\begin{tikzpicture}[auto,node distance=2cm,>=latex']
	\node [input] (x) {};
	\node [input, below of=x, node distance=1cm] (y) {};
	\node [block, right of=x] (ctrl) {\(G(s)\)};
	\node [sum, right of=ctrl] (sum) {};
	\node [output, right of=sum] (z) {};
	\draw [->] (x) -- node[]{\(X(s)\)} (ctrl);
	\draw [->] (ctrl) -- (sum);
	\draw [->] (y) -| node[pos=0.13]{\(Y(s)\)} (sum);
	\draw [->] (sum) -- node[]{\(Z(s)\)} (z);
\end{tikzpicture}\end{center}

Questo schema esprime la relazione \(Z(s) = X(s)G(s) + Y(s)\).
Dividendo per \(G(s)\) si ottiene
\(\frac{1}{G(s)}Z(s) = X(s) + \frac{Y(s)}{G(s)} \rightarrow
Z(s) = G(s) \bigl(X(s) + \frac{1}{G(s)}Y(s)\bigr)\)
che corrisponde all'equivalente
\begin{center}\begin{tikzpicture}[auto,node distance=2cm,>=latex']
	\node [input] (x) {};
	\node [sum, right of=x, node distance=3cm] (sum) {};
	\node [block, right of=sum] (ctrl1) {\(G(s)\)};
	\node [output, right of=ctrl1] (z) {};
	\draw [->] (x) -- node[pos=0.25]{\(X(s)\)} (sum);
	\draw [->] (sum) -- (ctrl1);
	\draw [->] (ctrl1) -- node[]{\(Z(s)\)} (z);
	% secondo livello
	\node [input, below of=x, node distance=1cm] (y) {};
	\node [block, right of=y] (ctrl2) {\(\frac{1}{G(s)}\)};
	\draw [->] (y) -- node[]{\(Y(s)\)} (ctrl2);
	\draw [->] (ctrl2) -| (sum);
\end{tikzpicture}\end{center}

\subsubsection{A valle di un blocco}
\begin{center}\begin{tikzpicture}[auto,node distance=2cm,>=latex']
	\node [input] (x) {};
	\node [input, below of=x, node distance=1cm] (y) {};
	\node [sum, right of=x] (sum) {};
	\node [block, right of=sum] (ctrl) {\(G(s)\)};
	\node [output, right of=ctrl] (z) {};
	\draw [->] (x) -- node[]{\(X(s)\)} (sum);
	\draw [->] (y) -| node[pos=0.2]{\(Y(s)\)} (sum);
	\draw [->] (sum) -- (ctrl) {};
	\draw [->] (ctrl) -- node[]{\(Z(s)\)} (z);
\end{tikzpicture}\end{center}

Questo schema esprime la relazione \(Z(s) = G(s)\bigl(X(s) + Y(s)\bigr)\),
che può essere reinterpretata, eseguendo il prodotto, come \(Z(s) = G(s)X(s) + G(s)Y(s)\).
Lo schema equivalente è il seguente
\begin{center}\begin{tikzpicture}[auto,node distance=2cm,>=latex']
	\node [input] (x) {};
	\node [input, below of=x] (y) {};
	\node [block, right of=x] (ctrl1) {\(G(s)\)};
	\node [block, below of=ctrl1] (ctrl2) {\(G(s)\)};
	\node [sum, right of=ctrl1, node distance=2cm] (sum) {};
	\node [output, right of=sum] (z) {};
	\draw [->] (x) -- node[]{\(X(s)\)} (ctrl1);
	\draw [->] (y) -- node[]{\(Y(s)\)} (ctrl2);
	\draw [->] (ctrl1) -- (sum);
	\draw [->] (ctrl2) -| (sum);
	\draw [->] (sum) -- node[]{\(Z(s)\)} (z);
\end{tikzpicture}\end{center}

\subsection{Spostamento nodo rispetto un sommatore}
\subsubsection{A monte di un sommatore}
\begin{center}\begin{tikzpicture}[auto,node distance=2cm,>=latex']
	\node [input] (x) {};
	\node [sum, right of=x] (sum) {};
	\node [input, above of=sum, node distance=1cm] (y) {};
	\node [output, right of=sum] (z) {};
	\draw [->] (x) -- node[]{\(X(s)\)} (sum);
	\draw [->] (y) -- node[]{\(Y(s)\)} (sum);
	\draw [->] (sum) -- node[pos=0.7]{\(Z(s)\)} (z);
	\draw (sum) -- node[name=out]{} (z);
	\node [output, below of=out, node distance=1cm] (zz) {};
	\draw [->] (out) -- (zz);
\end{tikzpicture}\end{center}

Questo schema presenta un nodo dove due output \(Z\) presentano lo stesso valore.
È possibile ottenere un equivalente separando i due output:
\begin{center}\begin{tikzpicture}[auto,node distance=2cm,>=latex']
	\node [input] (x) {};
	\node [sum, right of=x] (sum1) {};
	\node [input, above of=sum1] (y) {};
	\node [output, right of=sum1] (z) {};
	\node [output, below of=z, node distance=1cm] (zz) {};
	\node [sum, left of=zz] (sum2) {};
	\draw [->] (x) -- node[name=inx]{\(X(s)\)} (sum1);
	\draw [->] (inx) |- (sum2);
	\draw (y) -- node[pos=0.2]{\(Y(s)\)} (sum1);
	\draw [->] (y) -- node[name=iny]{} (sum1);
	\draw [->] (iny) -| (sum2);
	\draw [->] (sum1) -- node[pos=0.9]{\(Z(s)\)} (z);
	\draw [->] (sum2) -- node[pos=0.9]{\(Z(s)\)} (zz);
\end{tikzpicture}\end{center}

\subsubsection{A valle di un sommatore}
\begin{center}\begin{tikzpicture}[auto,node distance=2cm,>=latex']
	\node [input] (x) {};
	\node [sum, right of=x] (sum) {};
	\node [input, above of=sum, node distance=1cm] (y) {};
	\node [output, right of=sum] (z) {};
	\draw [->] (x) -- node[name=in]{\(X(s)\)} (sum);
	\draw [->] (y) -- node[]{\(Y(s)\)} (sum);
	\draw [->] (sum) -- node[pos=0.9]{\(Z(s)\)} (z);
	\node [output, below of=in, node distance=1cm] (xx) {};
	\draw [->] (in) -- (xx);
\end{tikzpicture}\end{center}

Questo schema presenta una uscita uguale a \(X(s)\), perciò se si volesse porre a
monte il sommatore, è necessario replicarlo per sottrarre \(Y(s)\) per quell'output.
\begin{center}\begin{tikzpicture}[auto,node distance=2cm,>=latex']
	\node [input] (x) {};
	\node [sum, right of=x] (sum1) {};
	\node [input, above of=sum1, node distance=1cm] (y) {};
	\node [output, right of=sum1] (z) {};
	\draw [->] (x) -- node[name=inx]{\(X(s)\)} (sum1);
	\draw (y) -- node[pos=0.1]{\(Y(s)\)} (sum1);
	\draw [->] (y) -- node[name=iny]{} (sum1);
	\draw [->] (sum1) -- node[pos=0.9]{\(Z(s)\)} (z);
	% secondo livello
	\node [output, below of=z, node distance=1cm] (xx) {};
	\node [sum, left of=xx] (sum2) {};
	\draw [->] (inx) |- (sum2);
	\draw [->] (iny) -| node[pos=0.9,anchor=east]{\(-\)} (sum2);
	\draw [->] (sum2) -- node[pos=0.8]{\(X(s)\)} (xx);
\end{tikzpicture}\end{center}

\subsection{Riduzione di un anello a retroazione negativa o positiva}
\begin{center}\begin{tikzpicture}[auto,node distance=2cm,>=latex']
	\node [input] (x) {};
	\node [sum, right of=x] (sum) {};
	\node [block, right of=sum] (ctrl) {\(G(s)\)};
	\node [block, below of=ctrl, node distance=1.5cm] (rectrl) {\(H(s)\)};
	\node [output, right of=ctrl] (y) {};
	\draw [->] (x) -- node[]{\(X(s)\)} (sum);
	\draw [->] (sum) -- node[]{\(E(s)\)} (ctrl);
	\draw [->] (ctrl) -- node[name=out]{\(Y(s)\)} (y);
	\draw [->] (out) |- (rectrl);
	\draw [->] (rectrl) -| node[pos=0.9,anchor=east]{\(\mp\)} (sum);
	\draw (rectrl) -| node[pos=0.7,anchor=west]{\(Z(s)\)} (sum);
\end{tikzpicture}\end{center}

Questo schema esprime una relazione di \emph{retroazione} negativa con le seguenti equazioni:
\[\begin{cases}
	Y(s) = G(s)E(s) \\
	E(s) = X(s) \mp Z(s) \\
	Z(s) = H(s)Y(s)
\end{cases}\]
Ricavando \(Y(s)\) è possibile ottenere la relazione equivalente
\begin{align*}
	E(s) &= X(s) \mp H(s)Y(s) \\
	Y(s) &= G(s)X(s) \mp G(s)H(s)Y(s) \\
	Y(s) \pm G(s)H(s)Y(s) &= G(s)H(s) \\
	Y(s) \bigl(1 \pm G(s)H(s)\bigr) &= G(s)X(s) \\
	\implies Y(s) &= \frac{G(s)X(s)}{1 \pm G(s)H(s)}
\end{align*}
e di conseguenza anche lo schema equivalente
\begin{center}\begin{tikzpicture}[auto,node distance=2cm,>=latex']
	\node [input] (x) {};
	\node [block, right of=x] (ctrl) {\(\frac{G(s)}{1 \pm G(s)H(s)}\)};
	\node [output, right of=ctrl] (y) {};
	\draw [->] (x) -- node[]{\(X(s)\)} (ctrl);
	\draw [->] (ctrl) -- node[]{\(Y(s)\)} (y);
\end{tikzpicture}\end{center}

\subsection{Esercizi}
\begin{esercizio}[Dati \(A=\frac{1}{2}\) e \(y=x\), determinare \(B\)] ~ % fixes paragraph spacing

\begin{center}\begin{tikzpicture}[auto,node distance=2cm,>=latex']
	\node [input] (u) {};
	\node [tmp, right of=u] (tmp1) {};
	\node [tmp, right of=tmp1] (tmp2) {};
	\node [sum, right of=tmp2, node distance=2cm] (sum) {};
	\node [block, above of=tmp2, node distance=1cm] (A) {\(A\)};
	\node [block, below of=tmp2, node distance=1cm] (B) {\(B\)};
	\node [sum, right of=A] (sumA) {};
	\node [sum, right of=B] (sumB) {};
	\node [output, right of=sum] (y) {};
	\draw (u) -- node[pos=0]{\(u\)} (tmp1);
	\draw [->] (tmp1) |- (A);
	\draw [->] (tmp1) |- (B);
	\draw (tmp1) -- (tmp2);
	\draw [->] (tmp2) -| node[pos=0.9,anchor=west]{\(-\)} (sumA);
	\draw [->] (tmp2) -| node[pos=0.9,anchor=west]{\(-\)} (sumB);
	\draw [->] (A) -- (sumA);
	\draw [->] (B) -- (sumB);
	\draw [->] (sumA) -| (sum);
	\draw [->] (sumB) -| (sum);
	\draw [->] (sum) -- node[pos=1]{\(y\)} (y);
\end{tikzpicture}\end{center}

Applicando la teoria, si ricava l'equazione \(y = Ax + Bx - x - x = x(A+B -2)\).
Sostituendo le condizioni, si ha \(\frac{1}{2} + B - 2 = 1\), ovvero
\[
	B = \frac{5}{2}
\]
\end{esercizio}

\begin{esercizio}[Dati \(A=2\), \(B=\frac{1}{2}\), \(y=3x\), determinare \(C\)] ~

\begin{center}\begin{tikzpicture}[auto,node distance=2cm,>=latex']
	\node [input] (x) {};
	\node [sum, right of=x] (sum) {};
	\node [block, right of=sum] (B) {\(B\)};
	\node [block, above of=B, node distance=1.5cm] (A) {\(A\)};
	\node [block, below of=B, node distance=1.5cm] (C) {\(C\)};
	\node [tmp, right of=B] (tmp) {};
	\node [output, right of=tmp] (y) {};
	\draw [->] (x) -- node[pos=0]{\(x\)} (sum);
	\draw [->] (sum) -- (B) -- (tmp) -- node[pos=1]{\(y\)} (y);
	\draw [->] (tmp) |- (A) -| node[pos=0.9,anchor=west]{\(-\)} (sum);
	\draw [->] (tmp) |- (C) -| node[pos=0.9,anchor=west]{\(-\)} (sum);
\end{tikzpicture}\end{center}

Riconoscendo che \(A \| C\), posso semplificare il blocco come segue:
\begin{center}\begin{tikzpicture}[auto,node distance=2cm,>=latex']
	\node [input] (x) {};
	\node [sum, right of=x] (sum) {};
	\node [block, right of=sum] (B) {\(B\)};
	\node [block, below of=B, node distance=1.5cm] (C) {\(A + C\)};
	\node [tmp, right of=B] (tmp) {};
	\node [output, right of=tmp] (y) {};
	\draw [->] (x) -- node[pos=0]{\(x\)} (sum);
	\draw [->] (sum) -- (B) -- (tmp) -- node[pos=1]{\(y\)} (y);
	\draw [->] (tmp) |- (C) -| node[pos=0.9,anchor=west]{\(-\)} (sum);
\end{tikzpicture}\end{center}

È chiaro che il sistema è a retroazione negativa. Quindi l'equazione del sistema è
\begin{align*}
	y = \frac{B}{1+(A+C)B}x \Rightarrow 3x &= \frac{\frac{1}{2}}{1+(2+C)\frac{1}{2}}x \\
	6 &= \frac{1}{1+(2+C)\frac{1}{2}} \\
	1+(2+C)\frac{1}{2} &= 2 + \frac{C}{2} = \frac{1}{6} \\
	C &= -\frac{11}{3}
\end{align*}
\end{esercizio}

\begin{esercizio}[Semplifica il seguente sistema a blocchi e ricava la funzione di trasferimento] ~

\begin{center}\begin{tikzpicture}[auto,node distance=2cm,>=latex']
	\node [input] (x) {};
	\node [sum, right of=x] (sum1) {};
	\node [block, right of=sum1] (A) {\(A\)};
	\node [block, right of=A] (B) {\(B\)};
	\node [block, right of=B] (C) {\(C\)};
	\node [output, right of=C] (y) {};
	\draw [->] (x) -- node[pos=0]{\(x\)} (sum1);
	\draw [->] (sum1) -- (A);
	\draw [->] (A) -- node[name=AB]{} (B);
	\draw [->] (B) -- node[name=BC]{} (C);
	\draw [->] (C) -- node[name=Cy]{} (y);
	\draw (C) -- node[pos=1]{\(y\)} (y);
	% secondo livello
	\node [sum, below of=AB] (sum2) {};
	\node [sum, below of=BC] (sum3) {};
	\draw [->] (Cy) |- (sum3);
	\draw [->] (sum3) -- node[pos=0.9,anchor=north]{\(-\)} (sum2);
	\draw [->] (sum2) -| node[pos=0.9,anchor=east]{\(-\)} (sum);
	\draw [->] (AB) -- (sum2);
	\draw [->] (BC) -- (sum3);
\end{tikzpicture}\end{center}

Una semplificazione ottimale del suddetto schema sarebbe portare i nodi,
con i rispettivi sommatori, a valle dei blocchi.
\begin{center}\begin{tikzpicture}[auto,node distance=2cm,>=latex']
	\begin{scope}
		\node [input] (x) {};
		\node [sum, right of=x] (sum1) {};
		\node [block, right of=sum1] (A) {\(A\)};
		\node [block, right of=A] (B) {\(B \cdot C\)};
		\node [output, right of=B, node distance=3cm] (y) {};
		\draw [->] (x) -- node[pos=0]{\(x\)} (sum1);
		\draw [->] (sum1) -- (A);
		\draw [->] (A) -- node[name=AB]{} (B);
		\draw [->] (B) -- node[pos=1]{\(y\)} (y);
		\draw (B) -- node[name=o1,pos=0.3]{} (y);
		\draw (B) -- node[name=o2,pos=0.6]{} (y);
		\node [block, below of=o1, node distance=1.5cm] (C) {\(\frac{1}{C}\)};
		\node [sum, below of=C] (sum2) {};
		\draw [->] (o1) -- (C);
		\draw [->] (C) -- (sum2);
		\draw [->] (o2) |- (sum2);
		\node [sum, below of=AB, node distance=2.5cm] (sum3) {};
		\draw [->] (sum2) -- node[pos=0.9,anchor=north]{\(-\)} (sum3);
		\draw [->] (AB) -- (sum3);
		\draw [->] (sum3) -| node[pos=0.9,anchor=east]{\(-\)} (sum1);
	\end{scope}
	\begin{scope}[xshift=8.5cm]
		\node [input] (x) {};
		\node [sum, right of=x] (sum1) {};
		\node [block, right of=sum1] (A) {\(A \cdot B \cdot C\)};
		\node [output, right of=A, node distance=4cm] (y) {};
		\draw [->] (x) -- node[pos=0]{\(x\)} (sum1);
		\draw [->] (sum1) -- (A);
		\draw [->] (A) -- node[pos=1]{\(y\)} (y);
		\draw (A) -- node[name=o1,pos=0.2]{} (y);
		\draw (A) -- node[name=o2,pos=0.6]{} (y);
		\draw (A) -- node[name=o3,pos=0.85]{} (y);
		\node [block, below of=o1] (B) {\(\frac{1}{BC}\)};
		\node [block, below of=o2] (C) {\(\frac{1}{C}\)};
		\node [sum, below of=B] (sum2) {};
		\node [sum, below of=C] (sum3) {};
		\draw [->] (o1) -- (B);
		\draw [->] (o2) -- (C);
		\draw [->] (o3) |- (sum3);
		\draw [->] (B) -- (sum2);
		\draw [->] (C) -- (sum3);
		\draw [->] (sum3) -- node[pos=0.9,anchor=north]{\(-\)} (sum2);
		\draw [->] (sum2) -| node[pos=0.9,anchor=east]{\(-\)} (sum1);
	\end{scope}
\end{tikzpicture}
\begin{tikzpicture}[auto,node distance=2cm,>=latex']
	\node [input] (x) {};
	\node [sum, right of=x] (sum) {};
	\node [block, right of=sum] (A) {\(A \cdot B \cdot C\)};
	\node [output, right of=A] (y) {};
	\draw [->] (x) -- node[pos=0]{\(x\)} (sum);
	\draw [->] (sum) -- (A);
	\draw [->] (A) -- node[pos=1]{\(y\)} (y);
	\draw (A) -- node[name=o]{} (y);
	\node [block, below of=A] (B) {\(\frac{1}{BC} + \frac{1}{C} + 1\)};
	\draw [->] (o) |- (B);
	\draw [->] (B) -| node[pos=0.9,anchor=east]{\(-\)} (sum);
\end{tikzpicture}\end{center}
che esprime l'equazione
\[
	y = \frac{ABC}{1 + \Bigl( \frac{1}{BC} + \frac{1}{C} + 1 \Bigr) ABC}
\]
\end{esercizio}

\begin{esercizio}[Semplifica il seguente sistema a blocchi e ricava la funzione di trasferimento] ~
\begin{center}\begin{tikzpicture}[auto,node distance=2cm,>=latex']
	\node [input] (x) {};
	\node [sum, right of=x, node distance=2cm] (sum1) {};
	\node [output, right of=sum1] (y) {};
	\draw [->] (x) -- node[name=o1,pos=0.4]{} (sum1);
	\draw (x) -- node[pos=0.1]{\(x\)} (sum1);
	\draw [->] (sum1) -- node[name=o2,pos=0.6]{} (y);
	\draw (sum1) -- node[pos=0.9]{\(y\)} (y);
	\node [block, above of=sum1, node distance=1.5cm] (B) {\(B\)};
	\node [block, left of=B, node distance=1.2cm] (A) {\(A\)};
	\node [block, right of=B, node distance=1.2cm] (C) {\(C\)};
	\node [sum, above of=B] (sum2) {};
	\draw [->] (sum1) -| (A);
	\draw [->] (A) |- (sum2);
	\draw [->] (sum1) -| (C);
	\draw [->] (C) |- node[pos=0.9,anchor=south]{\(-\)} (sum2);
	\draw [->] (sum2) -- (B);
	\draw [->] (B) -- (sum1);
\end{tikzpicture}\end{center}

È possibile semplificare lo schema portando il sommatore superiore a valle del
blocco \(B\). Di conseguenza si riconosce una doppia struttura che riconduce ad
un sistema a retroazione negativa:
\begin{center}\begin{tikzpicture}[auto,node distance=2cm,>=latex']
	\begin{scope}
		\node [input] (x) {};
		\node [sum, right of=x, node distance=2cm] (sum1) {};
		\node [output, right of=sum1] (y) {};
		\draw [->] (x) -- node[name=o1,pos=0.4]{} (sum1);
		\draw (x) -- node[pos=0.1]{\(x\)} (sum1);
		\draw [->] (sum1) -- node[name=o2,pos=0.6]{} (y);
		\draw (sum1) -- node[pos=0.9]{\(y\)} (y);
		\node [tmp, above of=sum1, node distance=1.5cm] (B) {\(B\)};
		\node [block, left of=B, node distance=1.2cm] (A) {\(AB\)};
		\node [block, right of=B, node distance=1.2cm] (C) {\(BC\)};
		\node [sum, above of=B] (sum2) {};
		\draw [->] (sum1) -| (A);
		\draw [->] (A) |- (sum2);
		\draw [->] (sum1) -| (C);
		\draw [->] (C) |- node[pos=0.9,anchor=south]{\(-\)} (sum2);
		\draw [->] (sum2) -- (sum1);
	\end{scope}
	\begin{scope}[xshift=5cm]
		\node [input] (x) {};
		\node [tmp, right of=x, node distance=1cm] (tmp1) {};
		\node [tmp, right of=tmp1, node distance=1cm] (tmp2) {};
		\node [sum, right of=tmp2] (sum1) {};
		\node [block, above of=tmp2, node distance=1cm] (A) {\(AB\)};
		\node [tmp, below of=tmp2, node distance=1cm] (tmp2b) {};
		\draw [->] (x) -- node[pos=0]{\(x\)} (tmp1) |- (A);
		\draw [->] (A) -| (sum1);
		\draw [->] (x) -- (tmp1) |- (tmp2b) -| (sum1);
		% seconda parte
		\node [sum, right of=sum1] (sum2) {};
		\node [output, right of=sum2, node distance=3cm] (y) {};
		\draw [->] (sum1) -- (sum2);
		\draw [->] (sum2) -- node[name=align,pos=0.4]{} (y);
		\draw (sum2) -- node[pos=1]{\(y\)} (y);
		\node [block, below of=align, node distance=1cm] (B) {\(BC\)};
		\draw [->] (B) -| node[pos=0.9,anchor=east]{\(-\)} (sum2);
		\node [tmp, left of=y, node distance=0.5cm] (tmp3) {};
		\draw [->] (tmp3) |- (B);
	\end{scope}
\end{tikzpicture}
\begin{tikzpicture}[auto,node distance=2cm,>=latex']
	\begin{scope}
		\node [input] (x) {};
		\node [block, right of=x] (A) {\(1 + AB\)};
		\node [sum, right of=A, node distance=2cm] (sum1) {};
		\draw [->] (x) -- node[pos=0]{\(x\)} (A);
		\draw [->] (A) -- (sum1);
		% seconda parte
		\node [output, right of=sum1, node distance=3cm] (y) {};
		\draw [->] (sum1) -- node[name=align,pos=0.4]{} (y);
		\draw (sum1) -- node[pos=1]{\(y\)} (y);
		\node [block, below of=align, node distance=1cm] (B) {\(BC\)};
		\draw [->] (B) -| node[pos=0.9,anchor=east]{\(-\)} (sum1);
		\node [tmp, left of=y, node distance=0.5cm] (tmp3) {};
		\draw [->] (tmp3) |- (B);
	\end{scope}
	\begin{scope}[xshift=8cm]
		\node [input] (x) {};
		\node [sum, right of=x] (sum) {};
		\node [block, right of=sum] (A) {\(1 + AB\)};
		\node [output, right of=A] (y) {};
		\node [block, below of=A] (B) {\(BC \frac{1}{1 + AB}\)};
		\draw [->] (x) -- node[pos=0]{\(x\)} (sum);
		\draw [->] (sum) -- (A);
		\draw [->] (A) -- node[name=retro]{} (y);
		\draw (A) -- node[pos=1]{\(y\)} (y);
		\draw [->] (retro) |- (B);
		\draw [->] (B) -| node[pos=0.9]{\(-\)} (sum);
	\end{scope}
\end{tikzpicture}
\end{center}

La semplificazione ci permette di ricavare intuitivamente la funzione di trasferimento
del sistema:
\[
	y = \frac{1 + AB}{1 + BC \frac{1}{1 + AB} (1 + AB)} = \frac{1 + AB}{1 + BC}
\]
\end{esercizio}

\begin{esercizio}[Determina la funzione di trasferimento del seguente sistema] ~

\begin{center}\begin{tikzpicture}[auto,node distance=2cm,>=latex']
	\node [input] (x) {};
	\node [sum, right of=x] (sum1) {};
	\node [block, right of=sum1, node distance=1.5cm] (G1) {\(G_1\)};
	\node [sum, right of=G1, node distance=1.5cm] (sum2) {};
	\node [input, above of=sum2, node distance=1cm] (d) {};
	\node [block, right of=sum2, node distance=1.5cm] (G2) {\(G_2\)};
	\node [output, right of=G2, node distance=1.5cm] (y) {};
	\node [block, below of=sum2, node distance=1cm] (H) {\(H\)};
	\draw [->] (x) -- node[pos=0,anchor=south]{\(x\)} (sum1);
	\draw [->] (sum1) -- (G1);
	\draw [->] (G1) -- (sum2);
	\draw [->] (d) -- node[pos=0,anchor=east]{\(d\)} (sum2);
	\draw [->] (sum2) -- (G2);
	\draw [->] (G2) -- node[pos=1,anchor=south]{\(y\)} (y);
	\draw (G2) -- node[name=retro]{} (y);
	\draw [->] (retro) |- (H);
	\draw [->] (H) -| (sum1);
\end{tikzpicture}\end{center}

È possibile considerare le entrate uno ad uno e sommare le funzioni di
trasferimento ottenute.

\paragraph{Risposta a \(x\)}
\begin{center}\begin{tikzpicture}[auto,node distance=2cm,>=latex']
	\node [input] (x) {};
	\node [sum, right of=x] (sum) {};
	\node [block, right of=sum, node distance=1.5cm] (G) {\(G_1 G_2\)};
	\node [output, right of=G] (y) {};
	\node [block, below of=G, node distance=1.5cm] (H) {\(H\)};
	\draw [->] (x) -- node[pos=0]{\(x\)} (sum);
	\draw [->] (sum) -- (G);
	\draw [->] (G) -- node[pos=1]{\(y\)} (y);
	\draw (G) -- node[name=retro]{} (y);
	\draw [->] (retro) |- (H);
	\draw [->] (H) -| (sum);
\end{tikzpicture}\end{center}

\[
	y_x = \frac{G_1 G_2}{1 + H G_1 G_2}
\]

\paragraph{Risposta a \(d\)}
\begin{center}\begin{tikzpicture}[auto,node distance=2cm,>=latex']
	\node [input] (d) {};
	\node [sum, right of=d] (sum) {};
	\node [block, right of=sum, node distance=1.5cm] (G) {\(G_2\)};
	\node [output, right of=G] (y) {};
	\node [block, below of=G, node distance=1.5cm] (H) {\(G_1 H\)};
	\draw [->] (d) -- node[pos=0]{\(d\)} (sum);
	\draw [->] (sum) -- (G);
	\draw [->] (G) -- node[pos=1]{\(y\)} (y);
	\draw (G) -- node[name=retro]{} (y);
	\draw [->] (retro) |- (H);
	\draw [->] (H) -| (sum);
\end{tikzpicture}\end{center}

\[
	y_d = \frac{G_2}{1 + G_1 G_2 H}
\]

Quindi la funzione di trasferimento dell'intero sistema è
\[
	y = y_x + y_d = \frac{G_2 ( 1 + G_1 )}{1 + G_1 G_2 H}
\]
\end{esercizio}

\begin{esercizio}[Determina la funzione di trasferimento del seguente sistema] ~

\begin{center}\begin{tikzpicture}[auto,node distance=2cm,>=latex']
	\node [input] (x) {};
	\node [sum, right of=x] (sum1) {};
	\node [sum, right of=sum1, fill=red!30] (sum2) {};
	\node [block, right of=sum1, node distance=2.5cm, fill=red!30] (G1) {\(G_1\)};
	\node [block, below of=G1, node distance=1.5cm, fill=red!30] (G3) {\(G_3\)};
	\node [block, below of=G3, node distance=1.5cm] (G4) {\(G_4\)};
	\node [tmp, right of=G1, node distance=1cm] (tmp1) {};
	\node [block, right of=tmp1, node distance=1cm] (G2) {\(G_2\)};
	\node [block, above of=G2, node distance=1.5cm] (G6) {\(G_6\)};
	\node [tmp, right of=G2, node distance=1cm] (tmp2) {};
	\node [block, right of=tmp2, node distance=1cm] (G5) {\(G_5\)};
	\node [sum, right of=G5] (sum3) {};
	\node [output, right of=sum3, node distance=1.5cm] (y) {};
	\draw [->] (x) -- node[pos=0]{\(x\)} (sum1);
	\draw [->] (sum1) -- (sum2);
	\draw [->] (sum2) -- (G1);
	\draw [->] (G1) -- (tmp1) -- (G2);
	\draw [->] (G2) -- (G5);
	\draw [->] (G5) -- (sum3);
	\draw [->] (sum3) -- node[pos=1]{\(y\)} (y);
	\draw [->] (tmp1) |- (G6);
	\draw [->] (G6) -| (sum3);
	\draw [->] (tmp1) |- (G3);
	\draw [->] (G3) -| (sum2);
	\draw [->] (tmp2) |- (G4);
	\draw [->] (G4) -| node[pos=0.9]{\(-\)} (sum1);
\end{tikzpicture}\end{center}

Il sistema è più complesso rispetto gli altri. Il metodo di risoluzione consiste
nel riconoscere le parti che possono essere semplificare, come quella appena
evidenziata che costituisce una retroazione positiva.

\begin{center}\begin{tikzpicture}[auto,node distance=2cm,>=latex']
	\node [input] (x) {};
	\node [sum, right of=x] (sum1) {};
	\node [block, right of=sum1] (G1) {\(\frac{G_1}{1 - G_1 G_3}\)};
	\node [block, below of=G1, node distance=1.5cm] (G4) {\(G_4\)};
	\node [tmp, right of=G1, node distance=1cm] (tmp1) {};
	\node [block, right of=tmp1, node distance=1cm, fill=red!30] (G2) {\(G_2\)};
	\node [block, above of=G2, node distance=1.5cm, fill=red!30] (G6) {\(G_6\)};
	\node [tmp, right of=G2, node distance=1cm] (tmp2) {};
	\node [block, right of=tmp2, node distance=1cm, fill=red!30] (G5) {\(G_5\)};
	\node [sum, right of=G5, fill=red!30] (sum3) {};
	\node [output, right of=sum3, node distance=1.5cm] (y) {};
	\draw [->] (x) -- node[pos=0]{\(x\)} (sum1);
	\draw [->] (sum1) -- (G1);
	\draw [->] (G1) -- (tmp1) -- (G2);
	\draw [->] (G2) -- (G5);
	\draw [->] (G5) -- (sum3);
	\draw [->] (sum3) -- node[pos=1]{\(y\)} (y);
	\draw [->] (tmp1) |- (G6);
	\draw [->] (G6) -| (sum3);
	\draw [->] (tmp2) |- (G4);
	\draw [->] (G4) -| node[pos=0.9]{\(-\)} (sum1);
\end{tikzpicture}\end{center}
È possibile, per l'area evidenziata, portare \(G_2\) a monte del nodo per il ramo
di \(G_6\).
\begin{center}\begin{tikzpicture}[auto,node distance=2cm,>=latex']
	\node [input] (x) {};
	\node [sum, right of=x] (sum1) {};
	\node [block, right of=sum1, fill=red!30] (G1) {\(\frac{G_1}{1 - G_1 G_3}\)};
	\node [block, below of=G1, node distance=1.5cm] (G4) {\(G_4\)};
	\node [block, right of=G1, fill=red!30] (G2) {\(G_2\)};
	\node [tmp, right of=G2, node distance=1cm] (tmp1) {};
	\node [block, right of=tmp1, node distance=1cm, fill=blue!30] (G5) {\(G_5\)};
	\node [block, above of=G5, node distance=1.5cm, fill=blue!30] (G6) {\(\frac{G_6}{G_2}\)};
	\node [sum, right of=G5, fill=blue!30] (sum3) {};
	\node [output, right of=sum3, node distance=1.5cm] (y) {};
	\draw [->] (x) -- node[pos=0]{\(x\)} (sum1);
	\draw [->] (sum1) -- (G1);
	\draw [->] (G1) -- (G2);
	\draw [->] (G2) -- (tmp1) -- (G5);
	\draw [->] (G5) -- (sum3);
	\draw [->] (sum3) -- node[pos=1]{\(y\)} (y);
	\draw [->] (tmp1) |- (G6);
	\draw [->] (G6) -| (sum3);
	\draw [->] (tmp1) |- (G4);
	\draw [->] (G4) -| node[pos=0.9]{\(-\)} (sum1);
\end{tikzpicture}\end{center}
Gli elementi evidenziati in rosso rappresentano una serie,
mentre quelli in blu sono paralleli.
\begin{center}\begin{tikzpicture}[auto,node distance=2cm,>=latex']
	\node [input] (x) {};
	\node [sum, right of=x, fill=red!30] (sum1) {};
	\node [block, right of=sum1, fill=red!30] (G1) {\(\frac{G_1 G_2}{1 - G_1 G_3}\)};
	\node [block, below of=G1, node distance=1.5cm, fill=red!30] (G4) {\(G_4\)};
	\node [tmp, right of=G1, node distance=1.5cm] (tmp1) {};
	\node [block, right of=tmp1, node distance=1.5cm] (G5) {\(G_5 + \frac{G_6}{G_2}\)};
	\node [output, right of=G5, node distance=1.5cm] (y) {};
	\draw [->] (x) -- node[pos=0]{\(x\)} (sum1);
	\draw [->] (sum1) -- (G1);
	\draw [->] (G1) -- (tmp1) -- (G5);
	\draw [->] (G5) -- node[pos=1]{\(y\)} (y);
	\draw [->] (tmp1) |- (G4);
	\draw [->] (G4) -| node[pos=0.9]{\(-\)} (sum1);
\end{tikzpicture}\end{center}
Infine si semplifica il sistema a retroazione negativa e si ricava la seguente equazione:
\[
	y = \frac{\frac{G_1 G_2}{1 - G_1 G_3}}{1 + G_4 \Bigl( \frac{G_1 G_2}{1 - G_1 G_3} \Bigr)} \Bigl( G_5 + \frac{G_6}{G_2} \Bigr) =
	\frac{\frac{G_1 G_2}{1 - G_1 G_3}}{\frac{1 - G_1 G_3 + G_4 G_1 G_2}{1 - G_1 G_3}} \Bigl( \frac{G_2 G_5 + G_6}{G_2} \Bigr) =
	\frac{G_1 \bigl( G_2 G_5 + G_6 \bigr)}{1 - G_1 \bigl( G_3 - G_2 G_4 \bigr)}
\]
\end{esercizio}
