% lingua
\usepackage[utf8]{inputenc}
\usepackage[T1]{fontenc}
\usepackage{lmodern}
\usepackage[italian]{babel}
\usepackage[babel]{csquotes}

\usepackage[mark]{gitinfo2}

% ambienti math
\usepackage{amsmath}
\usepackage{amsthm}
\usepackage{amsfonts}
\usepackage{amssymb}
\usepackage{thmtools}
\usepackage{cancel}
\usepackage{bm}

% indici acronimi bibliografia
\usepackage{makeidx}
%\usepackage[backend=biber]{biblatex}
\usepackage{multicol}
\usepackage{acronym}

% diagrammi circuiti elettrici
\usepackage[siunitx]{circuitikz}

% grafica
\usepackage{graphicx}
\usepackage{xcolor}
\usepackage{pgfplots,pgfplotstable}
\pgfplotsset{compat=1.12,/tikz/prefix=plots/}
\usetikzlibrary{math,matrix,chains}
\usetikzlibrary{scopes,positioning,fit,intersections}
\usetikzlibrary{angles,shapes,arrows,patterns,fadings}
\usetikzlibrary{decorations.pathreplacing,decorations.pathmorphing,decorations.markings,decorations.shapes}
\usetikzlibrary{tikzmark}
\usepgfplotslibrary{fillbetween,patchplots}
\pgfplotsset{every linear axis/.append style={axis lines=middle,enlargelimits}}
\pgfplotsset{trig format plots=rad}
\pgfplotsset{/pgfplots/colormap={graywhite}{gray=(0.75) gray=(1.0)}}

% oriented lines
% Thanks: http://tex.stackexchange.com/questions/163689/add-arrows-to-a-smooth-tikz-function
\tikzset{
    set arrow inside/.code={\pgfqkeys{/tikz/arrow inside}{#1}},
    set arrow inside={end/.initial=>, opt/.initial=},
    /pgf/decoration/Mark/.style={
        mark/.expanded=at position #1 with
        {
            \noexpand\arrow[\pgfkeysvalueof{/tikz/arrow inside/opt}]{\pgfkeysvalueof{/tikz/arrow inside/end}}
        }
    },
    arrow inside/.style 2 args={
        set arrow inside={#1},
        postaction={
            decorate,decoration={
                markings,Mark/.list={#2}
            }
        }
    },
}

% icone creative commons
\usepackage{ccicons}

% float and figure
\usepackage{float}
\usepackage{subfig}
\usepackage{caption}
\captionsetup{tableposition=top,figureposition=bottom,font=small,format=hang}
\usepackage{booktabs}
\usepackage{tablefootnote}
\renewcommand{\thefootnote}{\fnsymbol{footnote}}
\newcommand{\footnoteref}[1]{\textsuperscript{\ref{#1}}}

% hyperlink
\usepackage{hyperref}
\hypersetup{
	pdfauthor={Giovanni Grieco},
	pdftitle={Appunti di Fondamenti di Automatica},
	pdfsubject={Modulo I - Analisi di Sistemi di Controllo},
	pdfencoding=auto,
	psdextra,
	colorlinks,
	linkcolor={black},
	citecolor={blue!50!black},
	urlcolor={blue!80!black}
}

% definizioni nuovi stili teoremi
\theoremstyle{definition}
\newtheorem{definizione}{Definizione}[chapter]
\newtheorem{esempio}{Esempio}[chapter]
\newtheorem{esercizio}{Esercizio}[chapter]
\newtheorem{nota}{Nota}[chapter]
\newcommand{\keyword}[2][]{\textsc{#2}\index{#1}}

% fix: Warning Font shape for text `textbullet' undefined (Font)
\renewcommand\textbullet{\ensuremath{\bullet}}

% Simboli matematici di default
\renewcommand\epsilon{\varepsilon}
