\documentclass[10pt,a4paper,onecolumn,titlepage,twoside,openright,final]{book}
% lingua
\usepackage[utf8]{inputenc}
\usepackage[T1]{fontenc}
\usepackage{lmodern}
\usepackage[italian]{babel}
\usepackage[babel]{csquotes}

\usepackage[mark]{gitinfo2}

% ambienti math
\usepackage{amsmath}
\usepackage{amsthm}
\usepackage{amsfonts}
\usepackage{amssymb}
\usepackage{thmtools}

% indici acronimi bibliografia
\usepackage{makeidx}
%\usepackage[backend=biber]{biblatex}
\usepackage{multicol}
\usepackage{acronym}

% diagrammi circuiti elettrici
\usepackage[siunitx]{circuitikz}

% grafica
\usepackage{graphicx}
\usepackage{xcolor}
\usepackage{pgfplots,pgfplotstable}
\pgfplotsset{compat=1.12,/tikz/prefix=plots/}
\usetikzlibrary{math,matrix,chains}
\usetikzlibrary{scopes,positioning,fit,intersections}
\usetikzlibrary{angles,shapes,arrows,patterns,fadings}
\usetikzlibrary{decorations.pathreplacing,decorations.pathmorphing,decorations.markings,decorations.shapes}
\usetikzlibrary{tikzmark}
\usepgfplotslibrary{fillbetween,patchplots}
\pgfplotsset{every linear axis/.append style={axis lines=middle,no markers,enlargelimits}}
\pgfplotsset{trig format plots=rad}
\pgfplotsset{/pgfplots/colormap={graywhite}{gray=(0.75) gray=(1.0)}}

% icone creative commons
\usepackage{ccicons}

% float and figure
\usepackage{float}
\usepackage{subfig}
\usepackage{caption}
\captionsetup{tableposition=top,figureposition=bottom,font=small,format=hang}
\usepackage{booktabs}
\usepackage{tablefootnote}
\renewcommand{\thefootnote}{\fnsymbol{footnote}}
\newcommand{\footnoteref}[1]{\textsuperscript{\ref{#1}}}

% hyperlink
\usepackage{hyperref}
\hypersetup{
	pdfauthor={Giovanni Grieco},
	pdftitle={Appunti di Fondamenti di Automatica},
	pdfsubject={Modulo I - Analisi di Sistemi di Controllo},
	pdfencoding=auto,
	psdextra,
	colorlinks,
	linkcolor={black},
	citecolor={blue!50!black},
	urlcolor={blue!80!black}
}


% definizioni nuovi stili teoremi
\theoremstyle{definition}
\newtheorem{definizione}{Definizione}[chapter]
\newtheorem{esempio}{Esempio}[chapter]
\newtheorem{esercizio}{Esercizio}[chapter]
\newtheorem{nota}{Nota}[chapter]
\newcommand{\keyword}[2][]{\textsc{#2}\index{#1}}

% fix: Warning Font shape for text `textbullet' undefined (Font)
\renewcommand\textbullet{\ensuremath{\bullet}}


% formato pagina
\usepackage[width=15.00cm, left=3.00cm, height=23.00cm]{geometry}

% frontespizio Poliba
\usepackage{polibatitle}
\title{Appunti di \\ Fondamenti di Automatica}
\subtitle{Modulo I: \emph{Analisi di sistemi di controllo}}
\author{Giovanni Grieco}
\authoremail{giovanni.grc96@gmail.com}
\authorURL{https://it.linkedin.com/in/giovanni-grieco-17284459}
\professor{Prof.ssa M. Fanti}
\department{Dipartimento di Ingegneria Elettrica e dell'Informazione}
\version{\gitRel}
\versiondate{\gitAuthorDate}

\makeindex

\begin{document}

% diagrammi a blocchi
\tikzset{
	block/.style = {draw, fill=white, rectangle, minimum height=3em, minimum width=3em},
	tmp/.style  = {coordinate},
	sum/.style= {draw, fill=white, circle, node distance=1cm},
	input/.style = {coordinate},
	output/.style= {coordinate},
	pinstyle/.style = {pin edge={to-,thin,black}}
}

\frontmatter
\maketitle
\newpage

\vspace{\stretch{2}}\null
\vfill
Il presente documento è rilasciato sotto licenza \ccLogo\  \textbf{Creative Commons 4.0 by-sa-nc} \ccbyncsa.

È consentita la creazione di opere derivate, traduzioni, adattamenti, totali o parziali,
fatta salva l'attribuzione dell'autore originale e il mantenimento della licenza.

Per collaborare alla stesura e alla correzione dei sorgenti è possibile iscriversi
come contributori al repository GitHub all'indirizzo \url{https://github.com/corsaroquad/poliba-automatica}

Giovanni \textsc{Grieco}\\Politecnico di Bari
\cleardoublepage\clearpage{\pagestyle{empty}\cleardoublepage}

% indice contenuti
\tableofcontents

\mainmatter
\part{Analisi dei Sistemi di Controllo}
% TODO: Prima parte - prima dell'esonero di novembre
\chapter{Diagrammi a blocchi}

I diagrammi a blocchi servono per descrivere in modo astratto i sistemi automatici.
È possibile descrivere all'interno del blocco una propria \emph{funzione di trasferimento},
e questa sarà preceduta da delle \emph{frecce entranti} che faranno da \emph{input} e
da delle \emph{frecce uscenti} che faranno da \emph{output}.

Ogni input e output, secondo la logica del sistema, può essere sommato o sottrato con
altri fattori, come altri input e/o output oppure disturbi.
Il seguente esempio mostra un sistema statico con solo un input e un output
(altresì definito \emph{sistema SISO}).
\begin{center}\begin{tikzpicture}[auto, node distance=2cm,>=latex']
	\node [input, name=rinput] (rinput) {};
	\node [block, right of=rinput] (controller) {\(G(s)\)};
	\node [output, right of=controller, node distance=2cm] (output) {};
	\draw [->] (rinput) -- node{\(X(s)\)} (controller);
	\draw [->] (controller) -- node [name=y] {\(Y(s)\)}(output);
\end{tikzpicture}\end{center}

In caso di strutture più complesse e con l'interazione di più grandezze si parla
di \emph{sistema interconnesso}.

Inoltre esistono diverse relazioni che regolano la struttura dei diagrammi a blocchi.

\section{Strutture equivalenti}
\subsection{Blocchi in parallelo}
\begin{center}\begin{tikzpicture}[auto, node distance=2cm,>=latex']
	\node [input, name=xinput] (xinput) {};
	\node [block, right of=xinput] (ctrl1) {\(G_1(s)\)};
	\node [block, below of=ctrl1, node distance=2cm] (ctrl2) {\(G_2(s)\)};
	\node [sum, right of=ctrl1, node distance=2cm] (sum) {};
	\node [output, right of=sum] (output) {};
	\draw [->] (xinput) -- node[name=z]{} node{\(X(s)\)} (ctrl1);
	\draw [->] (ctrl1) -- node[name=out1] {\(Y_1(s)\)} (sum);
	\draw [->] (z) |- (ctrl2);
	\draw [->] (ctrl2) -| node[pos=0.25,anchor=south]{\(Y_2(s)\)} (sum);
	\draw [->] (sum) -- node[] {\(Y(s)\)} (output);
\end{tikzpicture}\end{center}

Le equazioni espresse nello schema sono
\begin{align*}
	Y_1(s) &= G_1(s)\,X(s) \\
	Y_2(s) &= G_2(s)\,X(s) \\
	Y(s) &= Y_1(s) + Y_2(s) \\
	\implies Y(s) &= \sum_{i=1}^\infty G_i(s)\,X(s)
\end{align*}
e quindi si può semplificare lo schema con un unico blocco che ne faccia la somma:
\begin{center}\begin{tikzpicture}[auto, node distance=2cm,>=latex']
	\node [input, name=x] (x) {};
	\node [block, right of=x] (ctrl) {\(\sum_{i=0}^\infty G_i(s)\)};
	\node [output, right of=ctrl] (y) {};
	\draw [->] (x) -- node[]{\(X(s)\)} (ctrl);
	\draw [->] (ctrl) -- node[]{\(Y(s)\)} (y);
\end{tikzpicture}\end{center}

Si conclude che \emph{il parallelo di due o più blocchi è equivalente alla somma
algebrica delle f.d.t. (guadagni) dei rispettivi blocchi interessati}.

\subsection{Blocchi in serie (o in cascata)}
\begin{center}\begin{tikzpicture}[auto, node distance=2cm,>=latex']
	\node [input, name=x] (x) {};
	\node [block, right of=x] (ctrl1) {\(G_1(s)\)};
	\node [block, right of=ctrl1, node distance=3cm] (ctrl2) {\(G_2(s)\)};
	\node [output, right of=ctrl2] (y) {};
	\draw [->] (x) -- node[]{\(X(s)\)} (ctrl1);
	\draw [->] (ctrl1) -- node[]{\(Z(s)\)} (ctrl2);
	\draw [->] (ctrl2) -- node[]{\(Y(s)\)} (y);
\end{tikzpicture}\end{center}

Le equazioni espresse nello schema sono
\begin{align*}
	Y(s) &= G_2(s)Z(s) \\
	Z(s) &= G_1(s)X(s) \\
	\implies Y(s) &= \prod_{i=1}^\infty G_i(s)X(s)
\end{align*}
e quindi si può semplificare lo schema con un blocco che ne faccia il prodotto:
\begin{center}\begin{tikzpicture}[auto, node distance=2cm,>=latex']
	\node [input] (x) {};
	\node [block, right of=x] (ctrl) {\(\prod_{i=0}^\infty G_i(s)\)};
	\node [output, right of=ctrl] (y) {};
	\draw [->] (x) -- node[]{\(X(s)\)} (ctrl);
	\draw [->] (ctrl) -- node[]{\(Y(s)\)} (y);
\end{tikzpicture}\end{center}

Si conclude che \emph{una serie di blocchi equivale al prodotto algebrico delle
f.d.t. dei rispettivi blocchi interessati}.

\subsection{Scambio di nodi sommatori}
\begin{center}\begin{tikzpicture}[auto, node distance=2cm,>=latex']
	\node [input] (x) {};
	\node [sum, right of=x] (sum1) {};
	\node [input, above of=sum1] (y) {};
	\node [sum, right of=sum1] (sum2) {};
	\node [input, above of=sum2] (w) {};
	\node [input, right of=sum2] (z) {};
	\draw [->] (x) -- node[]{\(X(s)\)} (sum1);
	\draw [->] (y) -- node[]{\(Y(s)\)} (sum1);
	\draw [->] (sum1) -- (sum2);
	\draw [->] (w) -- node[]{\(W(s)\)} (sum2);
	\draw [->] (sum2) -- node[]{\(Z(s)\)} (z);
\end{tikzpicture}\end{center}

L'equazione è \(Z(s) = W(s) + \bigl(Y(s) + X(s)\bigr)\) che, per la proprietà associativa
dell'addizione, può essere rivalutata come
\begin{align*}
	Z(s) &= \bigl(X(s) + W(s)\bigr) + Y(s) \\
	Z(s) &= \bigl(Y(s) + W(s)\bigr) + X(s) \\
	Z(s) &= X(s) + Y(s) + W(s)
\end{align*}
\begin{center}\begin{tikzpicture}[auto,node distance=2cm,>=latex']
	\begin{scope}
		\node [input] (x) {};
		\node [sum, right of=x] (sum1) {};
		\node [input, above of=sum1] (w) {};
		\node [sum, right of=sum1] (sum2) {};
		\node [input, below of=sum2] (y) {};
		\node [output, right of=sum2] (z) {};
		\draw [->] (x) -- node[]{\(X(s)\)} (sum1);
		\draw [->] (w) -- node[]{\(W(s)\)} (sum1);
		\draw [->] (sum1) -- (sum2);
		\draw [->] (y) -- node[]{\(Y(s)\)} (sum2);
		\draw [->] (sum2) -- node[]{\(Z(s)\)} (z);
	\end{scope}
	\begin{scope}[xshift=5cm]
		\node [input] (y) {};
		\node [sum, right of=y, node distance=2cm] (sum1) {};
		\node [input, below of=sum1] (w) {};
		\node [sum, right of=sum1] (sum2) {};
		\node [input, above of=sum2] (x) {};
		\node [output, right of=sum2] (z) {};
		\draw [->] (y) -- node[]{\(Y(s)\)} (sum1);
		\draw [->] (w) -- node[]{\(W(s)\)} (sum1);
		\draw [->] (sum1) -- (sum2);
		\draw [->] (x) -- node[]{\(X(s)\)} (sum2);
		\draw [->] (sum2) -- node[]{\(Z(s)\)} (z);
	\end{scope}
	\begin{scope}[xshift=7cm,yshift=-4cm]
		\node [input] (x) {};
		\node [sum, right of=x, node distance=2cm] (sum) {};
		\node [input, above of=sum] (y) {};
		\node [input, below of=sum] (w) {};
		\node [output, right of=sum] (z) {};
		\draw [->] (x) -- node[]{\(X(s)\)} (sum);
		\draw [->] (y) -- node[]{\(Y(s)\)} (sum);
		\draw [->] (w) -- node[]{\(W(s)\)} (sum);
		\draw [->] (sum) -- node[]{\(Z(s)\)} (z);
	\end{scope}
\end{tikzpicture}\end{center}

Si conclude che \emph{l'ordine dei blocchi collegati ad un sommatore è ininfluente
e possono essere collegati da un unico blocco sommatore}.

\subsection{Spostamento di un punto di prelievo rispetto a un blocco}
\subsubsection{A monte di un blocco}
\begin{center}\begin{tikzpicture}[auto,node distance=2cm,>=latex']
	\node [input] (x) {};
	\node [block, right of=x] (ctrl) {\(G(s)\)};
	\node [output, right of=ctrl] (y) {};
	\node [output, below of=x, node distance=1cm] (yy) {};
	\draw [->] (x) -- node[]{\(X(s)\)} (ctrl);
	\draw [->] (ctrl) -- node[name=out]{\(Y(s)\)} (y);
	\draw [->] (out) |- node[pos=0.8]{\(Y(s)\)} (yy);
\end{tikzpicture}\end{center}

Questo schema, dato che ha per entrambe le uscite equazione \(Y(s) = G(s)X(s)\)
può essere sostituito dall'equivalente
\begin{center}\begin{tikzpicture}[auto,node distance=2cm,>=latex']
	\node [input] (x) {};
	\node [block, right of=x, node distance=4cm] (ctrl1) {\(G(s)\)};
	\node [output, right of=ctrl1] (y) {\(Y(s)\)};
	\draw [->] (x) -- node[name=align,pos=0.5]{\(X(s)\)} (ctrl1);
	\draw [] (x) -- node[name=in,pos=0.85]{} (ctrl1);
	\draw [->] (ctrl1) -- node[]{\(Y(s)\)} (y);
	% secondo livello
	\node [block, below of=align] (ctrl2) {\(G(s)\)};
	\node [output, left of=ctrl2, node distance=1.8cm] (yy) {};
	\draw [->] (in) |- (ctrl2);
	\draw [->] (ctrl2) -- node[]{\(Y(s)\)} (yy);
\end{tikzpicture}\end{center}

\subsubsection{A valle di un blocco}
\begin{center}\begin{tikzpicture}[auto,node distance=2cm,>=latex']
	\node [input] (x) {};
	\node [block, right of=x, node distance=3cm] (ctrl) {\(G(s)\)};
	\node [output, right of=ctrl] (y) {};
	\draw [->] (x) -- node[name=in,pos=0.5]{\(X(s)\)} (ctrl);
	\draw [->] (ctrl) -- node[]{\(Y(s)\)} (y);
	% secondo livello
	\node [output, below of=x, node distance=1cm] (xx) {};
	\draw [->] (in) |- node[pos=0.7]{\(X(s)\)} (xx);
\end{tikzpicture}\end{center}

Questo schema, dato che ha per una uscita equazione \(Y(s) = G(s)X(s)\) mentre
per l'altra l'identità \(X(s) = X(s)\), può essere riscritta ponendo alla prima
equazione \(X(s) = \frac{1}{G(s)}Y(s)\). Di conseguenza lo schema equivalente è
\begin{center}\begin{tikzpicture}[auto,node distance=2cm,>=latex']
	\node [input] (x) {};
	\node [block, right of=x] (ctrl1) {\(G(s)\)};
	\node [output, right of=ctrl1] (y) {};
	\draw [->] (x) -- node[]{\(X(s)\)} (ctrl1);
	\draw [->] (ctrl1) -- node[name=out]{\(Y(s)\)} (y);
	% secondo livello
	\node [block, below of=ctrl1, node distance=1.5cm] (ctrl2) {\(\frac{1}{G(s)}\)};
	\node [output, left of=ctrl2] (xx) {};
	\draw [->] (out) |- (ctrl2);
	\draw [->] (ctrl2) -- node[]{\(X(s)\)} (xx);
\end{tikzpicture}\end{center}

\subsection{Spostamento di un nodo sommatore rispetto a un blocco}
\subsubsection{A monte di un blocco}
\begin{center}\begin{tikzpicture}[auto,node distance=2cm,>=latex']
	\node [input] (x) {};
	\node [input, below of=x, node distance=1cm] (y) {};
	\node [block, right of=x] (ctrl) {\(G(s)\)};
	\node [sum, right of=ctrl] (sum) {};
	\node [output, right of=sum] (z) {};
	\draw [->] (x) -- node[]{\(X(s)\)} (ctrl);
	\draw [->] (ctrl) -- (sum);
	\draw [->] (y) -| node[pos=0.13]{\(Y(s)\)} (sum);
	\draw [->] (sum) -- node[]{\(Z(s)\)} (z);
\end{tikzpicture}\end{center}

Questo schema esprime la relazione \(Z(s) = X(s)G(s) + Y(s)\).
Dividendo per \(G(s)\) si ottiene
\(\frac{1}{G(s)}Z(s) = X(s) + \frac{Y(s)}{G(s)} \rightarrow
Z(s) = G(s) \bigl(X(s) + \frac{1}{G(s)}Y(s)\bigr)\)
che corrisponde all'equivalente
\begin{center}\begin{tikzpicture}[auto,node distance=2cm,>=latex']
	\node [input] (x) {};
	\node [sum, right of=x, node distance=3cm] (sum) {};
	\node [block, right of=sum] (ctrl1) {\(G(s)\)};
	\node [output, right of=ctrl1] (z) {};
	\draw [->] (x) -- node[pos=0.25]{\(X(s)\)} (sum);
	\draw [->] (sum) -- (ctrl1);
	\draw [->] (ctrl1) -- node[]{\(Z(s)\)} (z);
	% secondo livello
	\node [input, below of=x, node distance=1cm] (y) {};
	\node [block, right of=y] (ctrl2) {\(\frac{1}{G(s)}\)};
	\draw [->] (y) -- node[]{\(Y(s)\)} (ctrl2);
	\draw [->] (ctrl2) -| (sum);
\end{tikzpicture}\end{center}

\subsubsection{A valle di un blocco}
\begin{center}\begin{tikzpicture}[auto,node distance=2cm,>=latex']
	\node [input] (x) {};
	\node [input, below of=x, node distance=1cm] (y) {};
	\node [sum, right of=x] (sum) {};
	\node [block, right of=sum] (ctrl) {\(G(s)\)};
	\node [output, right of=ctrl] (z) {};
	\draw [->] (x) -- node[]{\(X(s)\)} (sum);
	\draw [->] (y) -| node[pos=0.2]{\(Y(s)\)} (sum);
	\draw [->] (sum) -- (ctrl) {};
	\draw [->] (ctrl) -- node[]{\(Z(s)\)} (z);
\end{tikzpicture}\end{center}

Questo schema esprime la relazione \(Z(s) = G(s)\bigl(X(s) + Y(s)\bigr)\),
che può essere reinterpretata, eseguendo il prodotto, come \(Z(s) = G(s)X(s) + G(s)Y(s)\).
Lo schema equivalente è il seguente
\begin{center}\begin{tikzpicture}[auto,node distance=2cm,>=latex']
	\node [input] (x) {};
	\node [input, below of=x] (y) {};
	\node [block, right of=x] (ctrl1) {\(G(s)\)};
	\node [block, below of=ctrl1] (ctrl2) {\(G(s)\)};
	\node [sum, right of=ctrl1, node distance=2cm] (sum) {};
	\node [output, right of=sum] (z) {};
	\draw [->] (x) -- node[]{\(X(s)\)} (ctrl1);
	\draw [->] (y) -- node[]{\(Y(s)\)} (ctrl2);
	\draw [->] (ctrl1) -- (sum);
	\draw [->] (ctrl2) -| (sum);
	\draw [->] (sum) -- node[]{\(Z(s)\)} (z);
\end{tikzpicture}\end{center}

\subsection{Spostamento nodo rispetto un sommatore}
\subsubsection{A monte di un sommatore}
\begin{center}\begin{tikzpicture}[auto,node distance=2cm,>=latex']
	\node [input] (x) {};
	\node [sum, right of=x] (sum) {};
	\node [input, above of=sum, node distance=1cm] (y) {};
	\node [output, right of=sum] (z) {};
	\draw [->] (x) -- node[]{\(X(s)\)} (sum);
	\draw [->] (y) -- node[]{\(Y(s)\)} (sum);
	\draw [->] (sum) -- node[pos=0.7]{\(Z(s)\)} (z);
	\draw (sum) -- node[name=out]{} (z);
	\node [output, below of=out, node distance=1cm] (zz) {};
	\draw [->] (out) -- (zz);
\end{tikzpicture}\end{center}

Questo schema presenta un nodo dove due output \(Z\) presentano lo stesso valore.
È possibile ottenere un equivalente separando i due output:
\begin{center}\begin{tikzpicture}[auto,node distance=2cm,>=latex']
	\node [input] (x) {};
	\node [sum, right of=x] (sum1) {};
	\node [input, above of=sum1] (y) {};
	\node [output, right of=sum1] (z) {};
	\node [output, below of=z, node distance=1cm] (zz) {};
	\node [sum, left of=zz] (sum2) {};
	\draw [->] (x) -- node[name=inx]{\(X(s)\)} (sum1);
	\draw [->] (inx) |- (sum2);
	\draw (y) -- node[pos=0.2]{\(Y(s)\)} (sum1);
	\draw [->] (y) -- node[name=iny]{} (sum1);
	\draw [->] (iny) -| (sum2);
	\draw [->] (sum1) -- node[pos=0.9]{\(Z(s)\)} (z);
	\draw [->] (sum2) -- node[pos=0.9]{\(Z(s)\)} (zz);
\end{tikzpicture}\end{center}

\subsubsection{A valle di un sommatore}
\begin{center}\begin{tikzpicture}[auto,node distance=2cm,>=latex']
	\node [input] (x) {};
	\node [sum, right of=x] (sum) {};
	\node [input, above of=sum, node distance=1cm] (y) {};
	\node [output, right of=sum] (z) {};
	\draw [->] (x) -- node[name=in]{\(X(s)\)} (sum);
	\draw [->] (y) -- node[]{\(Y(s)\)} (sum);
	\draw [->] (sum) -- node[pos=0.9]{\(Z(s)\)} (z);
	\node [output, below of=in, node distance=1cm] (xx) {};
	\draw [->] (in) -- (xx);
\end{tikzpicture}\end{center}

Questo schema presenta una uscita uguale a \(X(s)\), perciò se si volesse porre a
monte il sommatore, è necessario replicarlo per sottrarre \(Y(s)\) per quell'output.
\begin{center}\begin{tikzpicture}[auto,node distance=2cm,>=latex']
	\node [input] (x) {};
	\node [sum, right of=x] (sum1) {};
	\node [input, above of=sum1, node distance=1cm] (y) {};
	\node [output, right of=sum1] (z) {};
	\draw [->] (x) -- node[name=inx]{\(X(s)\)} (sum1);
	\draw (y) -- node[pos=0.1]{\(Y(s)\)} (sum1);
	\draw [->] (y) -- node[name=iny]{} (sum1);
	\draw [->] (sum1) -- node[pos=0.9]{\(Z(s)\)} (z);
	% secondo livello
	\node [output, below of=z, node distance=1cm] (xx) {};
	\node [sum, left of=xx] (sum2) {};
	\draw [->] (inx) |- (sum2);
	\draw [->] (iny) -| node[pos=0.9,anchor=east]{\(-\)} (sum2);
	\draw [->] (sum2) -- node[pos=0.8]{\(X(s)\)} (xx);
\end{tikzpicture}\end{center}

\subsection{Riduzione di un anello a retroazione negativa o positiva}
\begin{center}\begin{tikzpicture}[auto,node distance=2cm,>=latex']
	\node [input] (x) {};
	\node [sum, right of=x] (sum) {};
	\node [block, right of=sum] (ctrl) {\(G(s)\)};
	\node [block, below of=ctrl, node distance=1.5cm] (rectrl) {\(H(s)\)};
	\node [output, right of=ctrl] (y) {};
	\draw [->] (x) -- node[]{\(X(s)\)} (sum);
	\draw [->] (sum) -- node[]{\(E(s)\)} (ctrl);
	\draw [->] (ctrl) -- node[name=out]{\(Y(s)\)} (y);
	\draw [->] (out) |- (rectrl);
	\draw [->] (rectrl) -| node[pos=0.9,anchor=east]{\(\mp\)} (sum);
	\draw (rectrl) -| node[pos=0.7,anchor=west]{\(Z(s)\)} (sum);
\end{tikzpicture}\end{center}

Questo schema esprime una relazione di \emph{retroazione} negativa con le seguenti equazioni:
\[\begin{cases}
	Y(s) = G(s)E(s) \\
	E(s) = X(s) \mp Z(s) \\
	Z(s) = H(s)Y(s)
\end{cases}\]
Ricavando \(Y(s)\) è possibile ottenere la relazione equivalente
\begin{align*}
	E(s) &= X(s) \mp H(s)Y(s) \\
	Y(s) &= G(s)X(s) \mp G(s)H(s)Y(s) \\
	Y(s) \pm G(s)H(s)Y(s) &= G(s)H(s) \\
	Y(s) \bigl(1 \pm G(s)H(s)\bigr) &= G(s)X(s) \\
	\implies Y(s) &= \frac{G(s)X(s)}{1 \pm G(s)H(s)}
\end{align*}
e di conseguenza anche lo schema equivalente
\begin{center}\begin{tikzpicture}[auto,node distance=2cm,>=latex']
	\node [input] (x) {};
	\node [block, right of=x] (ctrl) {\(\frac{G(s)}{1 \pm G(s)H(s)}\)};
	\node [output, right of=ctrl] (y) {};
	\draw [->] (x) -- node[]{\(X(s)\)} (ctrl);
	\draw [->] (ctrl) -- node[]{\(Y(s)\)} (y);
\end{tikzpicture}\end{center}

\section{Esercizi}
\exercise{}
Dati \(A=\frac{1}{2}\) e \(y=x\), determinare \(B\)

\begin{center}\begin{tikzpicture}[auto,node distance=2cm,>=latex']
	\node [input] (u) {};
	\node [tmp, right of=u] (tmp1) {};
	\node [tmp, right of=tmp1] (tmp2) {};
	\node [sum, right of=tmp2, node distance=2cm] (sum) {};
	\node [block, above of=tmp2, node distance=1cm] (A) {\(A\)};
	\node [block, below of=tmp2, node distance=1cm] (B) {\(B\)};
	\node [sum, right of=A] (sumA) {};
	\node [sum, right of=B] (sumB) {};
	\node [output, right of=sum] (y) {};
	\draw (u) -- node[pos=0]{\(x\)} (tmp1);
	\draw [->] (tmp1) |- (A);
	\draw [->] (tmp1) |- (B);
	\draw (tmp1) -- (tmp2);
	\draw [->] (tmp2) -| node[pos=0.9,anchor=west]{\(-\)} (sumA);
	\draw [->] (tmp2) -| node[pos=0.9,anchor=west]{\(-\)} (sumB);
	\draw [->] (A) -- (sumA);
	\draw [->] (B) -- (sumB);
	\draw [->] (sumA) -| (sum);
	\draw [->] (sumB) -| (sum);
	\draw [->] (sum) -- node[pos=1]{\(y\)} (y);
\end{tikzpicture}\end{center}

Applicando la teoria, si ricava l'equazione \(y = Ax + Bx - x - x = x(A+B -2)\).
Sostituendo le condizioni, si ha \(\frac{1}{2} + B - 2 = 1\), ovvero
\[
	B = \frac{5}{2}
\]


\exercise{}
Dati \(A=2\), \(B=\frac{1}{2}\), \(y=3x\), determinare \(C\)]

\begin{center}\begin{tikzpicture}[auto,node distance=2cm,>=latex']
	\node [input] (x) {};
	\node [sum, right of=x] (sum) {};
	\node [block, right of=sum] (B) {\(B\)};
	\node [block, above of=B, node distance=1.5cm] (A) {\(A\)};
	\node [block, below of=B, node distance=1.5cm] (C) {\(C\)};
	\node [tmp, right of=B] (tmp) {};
	\node [output, right of=tmp] (y) {};
	\draw [->] (x) -- node[pos=0]{\(x\)} (sum);
	\draw [->] (sum) -- (B) -- (tmp) -- node[pos=1]{\(y\)} (y);
	\draw [->] (tmp) |- (A) -| node[pos=0.9,anchor=west]{\(-\)} (sum);
	\draw [->] (tmp) |- (C) -| node[pos=0.9,anchor=west]{\(-\)} (sum);
\end{tikzpicture}\end{center}

Riconoscendo che \(A \| C\), posso semplificare il blocco come segue:
\begin{center}\begin{tikzpicture}[auto,node distance=2cm,>=latex']
	\node [input] (x) {};
	\node [sum, right of=x] (sum) {};
	\node [block, right of=sum] (B) {\(B\)};
	\node [block, below of=B, node distance=1.5cm] (C) {\(A + C\)};
	\node [tmp, right of=B] (tmp) {};
	\node [output, right of=tmp] (y) {};
	\draw [->] (x) -- node[pos=0]{\(x\)} (sum);
	\draw [->] (sum) -- (B) -- (tmp) -- node[pos=1]{\(y\)} (y);
	\draw [->] (tmp) |- (C) -| node[pos=0.9,anchor=west]{\(-\)} (sum);
\end{tikzpicture}\end{center}

È chiaro che il sistema è a retroazione negativa. Quindi l'equazione del sistema è
\begin{align*}
	y = \frac{B}{1+(A+C)B}x \Rightarrow 3x &= \frac{\frac{1}{2}}{1+(2+C)\frac{1}{2}}x \\
	6 &= \frac{1}{1+(2+C)\frac{1}{2}} \\
	1+(2+C)\frac{1}{2} &= 2 + \frac{C}{2} = \frac{1}{6} \\
	C &= -\frac{11}{3}
\end{align*}


\exercise{}
Semplifica il seguente sistema a blocchi e ricava la funzione di trasferimento

\begin{center}\begin{tikzpicture}[auto,node distance=2cm,>=latex']
	\node [input] (x) {};
	\node [sum, right of=x] (sum1) {};
	\node [block, right of=sum1] (A) {\(A\)};
	\node [block, right of=A] (B) {\(B\)};
	\node [block, right of=B] (C) {\(C\)};
	\node [output, right of=C] (y) {};
	\draw [->] (x) -- node[pos=0]{\(x\)} (sum1);
	\draw [->] (sum1) -- (A);
	\draw [->] (A) -- node[name=AB]{} (B);
	\draw [->] (B) -- node[name=BC]{} (C);
	\draw [->] (C) -- node[name=Cy]{} (y);
	\draw (C) -- node[pos=1]{\(y\)} (y);
	% secondo livello
	\node [sum, below of=AB] (sum2) {};
	\node [sum, below of=BC] (sum3) {};
	\draw [->] (Cy) |- (sum3);
	\draw [->] (sum3) -- node[pos=0.9,anchor=north]{\(-\)} (sum2);
	\draw [->] (sum2) -| node[pos=0.9,anchor=east]{\(-\)} (sum);
	\draw [->] (AB) -- (sum2);
	\draw [->] (BC) -- (sum3);
\end{tikzpicture}\end{center}

Una semplificazione ottimale del suddetto schema sarebbe portare i nodi,
con i rispettivi sommatori, a valle dei blocchi.
\begin{center}\begin{tikzpicture}[auto,node distance=2cm,>=latex']
	\node [input] (x) {};
	\node [sum, right of=x] (sum1) {};
	\node [block, right of=sum1] (A) {\(A\)};
	\node [block, right of=A] (B) {\(B \cdot C\)};
	\node [output, right of=B, node distance=3cm] (y) {};
	\draw [->] (x) -- node[pos=0]{\(x\)} (sum1);
	\draw [->] (sum1) -- (A);
	\draw [->] (A) -- node[name=AB]{} (B);
	\draw [->] (B) -- node[pos=1]{\(y\)} (y);
	\draw (B) -- node[name=o1,pos=0.3]{} (y);
	\draw (B) -- node[name=o2,pos=0.6]{} (y);
	\node [block, below of=o1, node distance=1.5cm] (C) {\(\frac{1}{C}\)};
	\node [sum, below of=C] (sum2) {};
	\draw [->] (o1) -- (C);
	\draw [->] (C) -- (sum2);
	\draw [->] (o2) |- (sum2);
	\node [sum, below of=AB, node distance=2.5cm] (sum3) {};
	\draw [->] (sum2) -- node[pos=0.9,anchor=north]{\(-\)} (sum3);
	\draw [->] (AB) -- (sum3);
	\draw [->] (sum3) -| node[pos=0.9,anchor=east]{\(-\)} (sum1);
\end{tikzpicture} \\
\begin{tikzpicture}[auto,node distance=2cm,>=latex']
	\node [input] (x) {};
	\node [sum, right of=x] (sum1) {};
	\node [block, right of=sum1] (A) {\(A \cdot B \cdot C\)};
	\node [output, right of=A, node distance=4cm] (y) {};
	\draw [->] (x) -- node[pos=0]{\(x\)} (sum1);
	\draw [->] (sum1) -- (A);
	\draw [->] (A) -- node[pos=1]{\(y\)} (y);
	\draw (A) -- node[name=o1,pos=0.2]{} (y);
	\draw (A) -- node[name=o2,pos=0.6]{} (y);
	\draw (A) -- node[name=o3,pos=0.85]{} (y);
	\node [block, below of=o1] (B) {\(\frac{1}{BC}\)};
	\node [block, below of=o2] (C) {\(\frac{1}{C}\)};
	\node [sum, below of=B] (sum2) {};
	\node [sum, below of=C] (sum3) {};
	\draw [->] (o1) -- (B);
	\draw [->] (o2) -- (C);
	\draw [->] (o3) |- (sum3);
	\draw [->] (B) -- (sum2);
	\draw [->] (C) -- (sum3);
	\draw [->] (sum3) -- node[pos=0.9,anchor=north]{\(-\)} (sum2);
	\draw [->] (sum2) -| node[pos=0.9,anchor=east]{\(-\)} (sum1);
\end{tikzpicture} \\
\begin{tikzpicture}[auto,node distance=2cm,>=latex']
	\node [input] (x) {};
	\node [sum, right of=x] (sum) {};
	\node [block, right of=sum] (A) {\(A \cdot B \cdot C\)};
	\node [output, right of=A] (y) {};
	\draw [->] (x) -- node[pos=0]{\(x\)} (sum);
	\draw [->] (sum) -- (A);
	\draw [->] (A) -- node[pos=1]{\(y\)} (y);
	\draw (A) -- node[name=o]{} (y);
	\node [block, below of=A] (B) {\(\frac{1}{BC} - \frac{1}{C} - 1\)};
	\draw [->] (o) |- (B);
	\draw [->] (B) -| node[pos=0.9,anchor=east]{\(-\)} (sum);
\end{tikzpicture}\end{center}
che esprime l'equazione
\[
	y = \frac{ABC}{1 + \Bigl( \frac{1}{BC} - \frac{1}{C} - 1 \Bigr) ABC}
\]


\exercise{}
Semplifica il seguente sistema a blocchi e ricava la funzione di trasferimento
\begin{center}\begin{tikzpicture}[auto,node distance=2cm,>=latex']
	\node [input] (x) {};
	\node [sum, right of=x, node distance=2cm] (sum1) {};
	\node [output, right of=sum1] (y) {};
	\draw [->] (x) -- node[name=o1,pos=0.4]{} (sum1);
	\draw (x) -- node[pos=0.1]{\(x\)} (sum1);
	\draw [->] (sum1) -- node[name=o2,pos=0.6]{} (y);
	\draw (sum1) -- node[pos=0.9]{\(y\)} (y);
	\node [block, above of=sum1, node distance=1.5cm] (B) {\(B\)};
	\node [block, left of=B, node distance=1.2cm] (A) {\(A\)};
	\node [block, right of=B, node distance=1.2cm] (C) {\(C\)};
	\node [sum, above of=B] (sum2) {};
	\draw [->] (sum1) -| (A);
	\draw [->] (A) |- (sum2);
	\draw [->] (sum1) -| (C);
	\draw [->] (C) |- node[pos=0.9,anchor=south]{\(-\)} (sum2);
	\draw [->] (sum2) -- (B);
	\draw [->] (B) -- (sum1);
\end{tikzpicture}\end{center}

È possibile semplificare lo schema portando il sommatore superiore a valle del
blocco \(B\). Di conseguenza si riconosce una doppia struttura che riconduce ad
un sistema a retroazione negativa:
\begin{center}\begin{tikzpicture}[auto,node distance=2cm,>=latex']
	\begin{scope}
		\node [input] (x) {};
		\node [sum, right of=x, node distance=2cm] (sum1) {};
		\node [output, right of=sum1] (y) {};
		\draw [->] (x) -- node[name=o1,pos=0.4]{} (sum1);
		\draw (x) -- node[pos=0.1]{\(x\)} (sum1);
		\draw [->] (sum1) -- node[name=o2,pos=0.6]{} (y);
		\draw (sum1) -- node[pos=0.9]{\(y\)} (y);
		\node [tmp, above of=sum1, node distance=1.5cm] (B) {\(B\)};
		\node [block, left of=B, node distance=1.2cm] (A) {\(AB\)};
		\node [block, right of=B, node distance=1.2cm] (C) {\(BC\)};
		\node [sum, above of=B] (sum2) {};
		\draw [->] (sum1) -| (A);
		\draw [->] (A) |- (sum2);
		\draw [->] (sum1) -| (C);
		\draw [->] (C) |- node[pos=0.9,anchor=south]{\(-\)} (sum2);
		\draw [->] (sum2) -- (sum1);
	\end{scope}
	\begin{scope}[xshift=5cm]
		\node [input] (x) {};
		\node [tmp, right of=x, node distance=1cm] (tmp1) {};
		\node [tmp, right of=tmp1, node distance=1cm] (tmp2) {};
		\node [sum, right of=tmp2] (sum1) {};
		\node [block, above of=tmp2, node distance=1cm] (A) {\(AB\)};
		\node [tmp, below of=tmp2, node distance=1cm] (tmp2b) {};
		\draw [->] (x) -- node[pos=0]{\(x\)} (tmp1) |- (A);
		\draw [->] (A) -| (sum1);
		\draw [->] (x) -- (tmp1) |- (tmp2b) -| (sum1);
		% seconda parte
		\node [sum, right of=sum1] (sum2) {};
		\node [output, right of=sum2, node distance=3cm] (y) {};
		\draw [->] (sum1) -- (sum2);
		\draw [->] (sum2) -- node[name=align,pos=0.4]{} (y);
		\draw (sum2) -- node[pos=1]{\(y\)} (y);
		\node [block, below of=align, node distance=1cm] (B) {\(BC\)};
		\draw [->] (B) -| node[pos=0.9,anchor=east]{\(-\)} (sum2);
		\node [tmp, left of=y, node distance=0.5cm] (tmp3) {};
		\draw [->] (tmp3) |- (B);
	\end{scope}
\end{tikzpicture}
\begin{tikzpicture}[auto,node distance=2cm,>=latex']
	\begin{scope}
		\node [input] (x) {};
		\node [block, right of=x] (A) {\(1 + AB\)};
		\node [sum, right of=A, node distance=2cm] (sum1) {};
		\draw [->] (x) -- node[pos=0]{\(x\)} (A);
		\draw [->] (A) -- (sum1);
		% seconda parte
		\node [output, right of=sum1, node distance=3cm] (y) {};
		\draw [->] (sum1) -- node[name=align,pos=0.4]{} (y);
		\draw (sum1) -- node[pos=1]{\(y\)} (y);
		\node [block, below of=align, node distance=1cm] (B) {\(BC\)};
		\draw [->] (B) -| node[pos=0.9,anchor=east]{\(-\)} (sum1);
		\node [tmp, left of=y, node distance=0.5cm] (tmp3) {};
		\draw [->] (tmp3) |- (B);
	\end{scope}
	\begin{scope}[xshift=8cm]
		\node [input] (x) {};
		\node [sum, right of=x] (sum) {};
		\node [block, right of=sum] (A) {\(1 + AB\)};
		\node [output, right of=A] (y) {};
		\node [block, below of=A] (B) {\(BC \frac{1}{1 + AB}\)};
		\draw [->] (x) -- node[pos=0]{\(x\)} (sum);
		\draw [->] (sum) -- (A);
		\draw [->] (A) -- node[name=retro]{} (y);
		\draw (A) -- node[pos=1]{\(y\)} (y);
		\draw [->] (retro) |- (B);
		\draw [->] (B) -| node[pos=0.9]{\(-\)} (sum);
	\end{scope}
\end{tikzpicture}
\end{center}

La semplificazione ci permette di ricavare intuitivamente la funzione di trasferimento
del sistema:
\[
	y = \frac{1 + AB}{1 + BC \frac{1}{1 + AB} (1 + AB)} = \frac{1 + AB}{1 + BC}
\]


\exercise{}
Determina la funzione di trasferimento del seguente sistema
\begin{center}\begin{tikzpicture}[auto,node distance=2cm,>=latex']
	\node [input] (x) {};
	\node [sum, right of=x] (sum1) {};
	\node [block, right of=sum1, node distance=1.5cm] (G1) {\(G_1\)};
	\node [sum, right of=G1, node distance=1.5cm] (sum2) {};
	\node [input, above of=sum2, node distance=1cm] (d) {};
	\node [block, right of=sum2, node distance=1.5cm] (G2) {\(G_2\)};
	\node [output, right of=G2, node distance=1.5cm] (y) {};
	\node [block, below of=sum2, node distance=1cm] (H) {\(H\)};
	\draw [->] (x) -- node[pos=0,anchor=south]{\(x\)} (sum1);
	\draw [->] (sum1) -- (G1);
	\draw [->] (G1) -- (sum2);
	\draw [->] (d) -- node[pos=0,anchor=east]{\(d\)} (sum2);
	\draw [->] (sum2) -- (G2);
	\draw [->] (G2) -- node[pos=1,anchor=south]{\(y\)} (y);
	\draw (G2) -- node[name=retro]{} (y);
	\draw [->] (retro) |- (H);
	\draw [->] (H) -| (sum1);
\end{tikzpicture}\end{center}

È possibile considerare le entrate uno ad uno e sommare le funzioni di
trasferimento ottenute.

\paragraph{Risposta a \(x\)}
\begin{center}\begin{tikzpicture}[auto,node distance=2cm,>=latex']
	\node [input] (x) {};
	\node [sum, right of=x] (sum) {};
	\node [block, right of=sum, node distance=1.5cm] (G) {\(G_1 G_2\)};
	\node [output, right of=G] (y) {};
	\node [block, below of=G, node distance=1.5cm] (H) {\(H\)};
	\draw [->] (x) -- node[pos=0]{\(x\)} (sum);
	\draw [->] (sum) -- (G);
	\draw [->] (G) -- node[pos=1]{\(y\)} (y);
	\draw (G) -- node[name=retro]{} (y);
	\draw [->] (retro) |- (H);
	\draw [->] (H) -| (sum);
\end{tikzpicture}\end{center}

\[
	y_x = \frac{G_1 G_2}{1 - H G_1 G_2}
\]

\paragraph{Risposta a \(d\)}
\begin{center}\begin{tikzpicture}[auto,node distance=2cm,>=latex']
	\node [input] (d) {};
	\node [sum, right of=d] (sum) {};
	\node [block, right of=sum, node distance=1.5cm] (G) {\(G_2\)};
	\node [output, right of=G] (y) {};
	\node [block, below of=G, node distance=1.5cm] (H) {\(G_1 H\)};
	\draw [->] (d) -- node[pos=0]{\(d\)} (sum);
	\draw [->] (sum) -- (G);
	\draw [->] (G) -- node[pos=1]{\(y\)} (y);
	\draw (G) -- node[name=retro]{} (y);
	\draw [->] (retro) |- (H);
	\draw [->] (H) -| (sum);
\end{tikzpicture}\end{center}

\[
	y_d = \frac{G_2}{1 - G_1 G_2 H}
\]

Quindi la funzione di trasferimento dell'intero sistema è
\[
	y = y_x + y_d = \frac{G_2 ( 1 + G_1 )}{1 - G_1 G_2 H}
\]


\exercise{}
Determina la funzione di trasferimento del seguente sistema
\begin{center}\begin{tikzpicture}[auto,node distance=2cm,>=latex']
	\node [input] (x) {};
	\node [sum, right of=x] (sum1) {};
	\node [sum, right of=sum1, fill=red!30] (sum2) {};
	\node [block, right of=sum1, node distance=2.5cm, fill=red!30] (G1) {\(G_1\)};
	\node [block, below of=G1, node distance=1.5cm, fill=red!30] (G3) {\(G_3\)};
	\node [block, below of=G3, node distance=1.5cm] (G4) {\(G_4\)};
	\node [tmp, right of=G1, node distance=1cm] (tmp1) {};
	\node [block, right of=tmp1, node distance=1cm] (G2) {\(G_2\)};
	\node [block, above of=G2, node distance=1.5cm] (G6) {\(G_6\)};
	\node [tmp, right of=G2, node distance=1cm] (tmp2) {};
	\node [block, right of=tmp2, node distance=1cm] (G5) {\(G_5\)};
	\node [sum, right of=G5] (sum3) {};
	\node [output, right of=sum3, node distance=1.5cm] (y) {};
	\draw [->] (x) -- node[pos=0]{\(x\)} (sum1);
	\draw [->] (sum1) -- (sum2);
	\draw [->] (sum2) -- (G1);
	\draw [->] (G1) -- (tmp1) -- (G2);
	\draw [->] (G2) -- (G5);
	\draw [->] (G5) -- (sum3);
	\draw [->] (sum3) -- node[pos=1]{\(y\)} (y);
	\draw [->] (tmp1) |- (G6);
	\draw [->] (G6) -| (sum3);
	\draw [->] (tmp1) |- (G3);
	\draw [->] (G3) -| (sum2);
	\draw [->] (tmp2) |- (G4);
	\draw [->] (G4) -| node[pos=0.9]{\(-\)} (sum1);
\end{tikzpicture}\end{center}

Il sistema è più complesso rispetto gli altri. Il metodo di risoluzione consiste
nel riconoscere le parti che possono essere semplificare, come quella appena
evidenziata che costituisce una retroazione positiva.

\begin{center}\begin{tikzpicture}[auto,node distance=2cm,>=latex']
	\node [input] (x) {};
	\node [sum, right of=x] (sum1) {};
	\node [block, right of=sum1] (G1) {\(\frac{G_1}{1 - G_1 G_3}\)};
	\node [block, below of=G1, node distance=1.5cm] (G4) {\(G_4\)};
	\node [tmp, right of=G1, node distance=1cm] (tmp1) {};
	\node [block, right of=tmp1, node distance=1cm, fill=red!30] (G2) {\(G_2\)};
	\node [block, above of=G2, node distance=1.5cm, fill=red!30] (G6) {\(G_6\)};
	\node [tmp, right of=G2, node distance=1cm] (tmp2) {};
	\node [block, right of=tmp2, node distance=1cm, fill=red!30] (G5) {\(G_5\)};
	\node [sum, right of=G5, fill=red!30] (sum3) {};
	\node [output, right of=sum3, node distance=1.5cm] (y) {};
	\draw [->] (x) -- node[pos=0]{\(x\)} (sum1);
	\draw [->] (sum1) -- (G1);
	\draw [->] (G1) -- (tmp1) -- (G2);
	\draw [->] (G2) -- (G5);
	\draw [->] (G5) -- (sum3);
	\draw [->] (sum3) -- node[pos=1]{\(y\)} (y);
	\draw [->] (tmp1) |- (G6);
	\draw [->] (G6) -| (sum3);
	\draw [->] (tmp2) |- (G4);
	\draw [->] (G4) -| node[pos=0.9]{\(-\)} (sum1);
\end{tikzpicture}\end{center}
È possibile, per l'area evidenziata, portare \(G_2\) a monte del nodo per il ramo
di \(G_6\).
\begin{center}\begin{tikzpicture}[auto,node distance=2cm,>=latex']
	\node [input] (x) {};
	\node [sum, right of=x] (sum1) {};
	\node [block, right of=sum1, fill=red!30] (G1) {\(\frac{G_1}{1 - G_1 G_3}\)};
	\node [block, below of=G1, node distance=1.5cm] (G4) {\(G_4\)};
	\node [block, right of=G1, fill=red!30] (G2) {\(G_2\)};
	\node [tmp, right of=G2, node distance=1cm] (tmp1) {};
	\node [block, right of=tmp1, node distance=1cm, fill=blue!30] (G5) {\(G_5\)};
	\node [block, above of=G5, node distance=1.5cm, fill=blue!30] (G6) {\(\frac{G_6}{G_2}\)};
	\node [sum, right of=G5, fill=blue!30] (sum3) {};
	\node [output, right of=sum3, node distance=1.5cm] (y) {};
	\draw [->] (x) -- node[pos=0]{\(x\)} (sum1);
	\draw [->] (sum1) -- (G1);
	\draw [->] (G1) -- (G2);
	\draw [->] (G2) -- (tmp1) -- (G5);
	\draw [->] (G5) -- (sum3);
	\draw [->] (sum3) -- node[pos=1]{\(y\)} (y);
	\draw [->] (tmp1) |- (G6);
	\draw [->] (G6) -| (sum3);
	\draw [->] (tmp1) |- (G4);
	\draw [->] (G4) -| node[pos=0.9]{\(-\)} (sum1);
\end{tikzpicture}\end{center}
Gli elementi evidenziati in rosso rappresentano una serie,
mentre quelli in blu sono paralleli.
\begin{center}\begin{tikzpicture}[auto,node distance=2cm,>=latex']
	\node [input] (x) {};
	\node [sum, right of=x, fill=red!30] (sum1) {};
	\node [block, right of=sum1, fill=red!30] (G1) {\(\frac{G_1 G_2}{1 - G_1 G_3}\)};
	\node [block, below of=G1, node distance=1.5cm, fill=red!30] (G4) {\(G_4\)};
	\node [tmp, right of=G1, node distance=1.5cm] (tmp1) {};
	\node [block, right of=tmp1, node distance=1.5cm] (G5) {\(G_5 + \frac{G_6}{G_2}\)};
	\node [output, right of=G5, node distance=1.5cm] (y) {};
	\draw [->] (x) -- node[pos=0]{\(x\)} (sum1);
	\draw [->] (sum1) -- (G1);
	\draw [->] (G1) -- (tmp1) -- (G5);
	\draw [->] (G5) -- node[pos=1]{\(y\)} (y);
	\draw [->] (tmp1) |- (G4);
	\draw [->] (G4) -| node[pos=0.9]{\(-\)} (sum1);
\end{tikzpicture}\end{center}
Infine si semplifica il sistema a retroazione negativa e si ricava la seguente equazione:
\[
	y = \frac{\frac{G_1 G_2}{1 - G_1 G_3}}{1 + G_4 \Bigl( \frac{G_1 G_2}{1 - G_1 G_3} \Bigr)} \Bigl( G_5 + \frac{G_6}{G_2} \Bigr) =
	\frac{\frac{G_1 G_2}{1 - G_1 G_3}}{\frac{1 - G_1 G_3 + G_4 G_1 G_2}{1 - G_1 G_3}} \Bigl( \frac{G_2 G_5 + G_6}{G_2} \Bigr) =
	\frac{G_1 \bigl( G_2 G_5 + G_6 \bigr)}{1 - G_1 \bigl( G_3 - G_2 G_4 \bigr)}
\]


\chapter{Criterio di Routh}

Per un sistema è importante il fattore dell'\emph{accuratezza}, questo considerato
quando si considerano i poli del sistema e si analizzano il loro comportamento,
specie se il sistema fosse a \emph{retroazione} (o anello chiuso) e avesse possibili
disturbi/parametri che potrebbero variare variare.

Il \emph{luogo delle radici} è la descrizione dei valori dei poli che assumono al variare
dei parametri di sistema. Con la sua osservazione su un piano di Gauss, è possibile
determinare la stabilità del sistema e quindi comprendere il range possibile dei
parametri per un suo uso in sicurezza.

Per il tracciamento del luogo delle radici, si preferisce studiare il sistema
\emph{al limite della stabilità}, quindi è importante conoscere possibili poli
sull'asse immaginario/punto di origine e sul semipiano positivo. Poiché tali
punti corrispondono al limite della stabilità di un sistema in retroazione, per
determinarli si può far ricorso al \emph{criterio di Routh}.

È consigliato limitare il guadagno parametrico del sistema, questo perché si
possono verificare problemi di saturazione e i poli tendono verso il semipiano
destro (ovvero verso gli asintoti, quindi al limite della stabilità), quindi
l'intero sistema diventa instabile. L'obiettivo è comunque scegliere dei valori
parametrici tale da migliore il più possibile la precisione del sistema a regime.

Per applicare il criterio di Routh si usa la \emph{tabella di taratura} del luogo,
che consente di determinare il miglior guadagno parametrico e la corrispondente
stabilità del sistema.

\paragraph{Lemma di Routh}
Condizione necessaria affinché tutte le radici del polinomio \(q(s) = 0\) siano a
parte reale \(\Re s_i < 0\) è che i coefficienti \(a_i > 0\).

\begin{esempio}[Caso generale]
Sia dato il sistema generico
\[
	G(s) = \frac{b_n s^n + \dots + b_0}{a_n s^n + a_{n-1}s^{n-1} + \dots + a_1 s + a_0}
\]
Se si trovassero le radici del denominatore con \(a_i > 0\), il sistema è
sicuramente stabile.
La condizione è sufficiente per \(n=1\) e \(n=2\):
\begin{itemize}
	\item con \(a_1 s + a_0 = 0 \colon s = -\frac{a_0}{a_1}\)
	\item con \(a_2 s^2 + a_1 s + a_0 = 0 \colon s_{1,2} = -\frac{a_1 \pm \sqrt{a^2_1 -4a_0 a_2}}{2a_2}\)
\end{itemize}
Per sistemi più grandi si usa la \emph{tabella di Routh}:
\[\begin{array}{r|rrr}
	s^n 	& a_n 	  & a_{n-2} & \dots	\\
	s^{n-1} & a_{n-1} & a_{n-3} & \dots 	\\
	s^{n-2} & b_1 	  & b_2     & \dots 	\\
	\vdots 	& c_1
\end{array}\]
dove
\[
	b_1 = \frac{a_{n-1} a_{n-2} - a_n a_{n-3}}{a_{n-1}} \qquad
	c_1 = \frac{b_1 a_{n-3} - a_{n-1}b_2}{b_1}
\]

\end{esempio}

\begin{esempio}
Si considera il seguente sistema:
\[
	G(s) = \frac{s^5 + 3}{s^6 + 2s^5 + 2s^4 - s^3 - 2s - 2}
\]
Per verificare il criterio di Routh bisogna considerare la tabella di Routh e
seguire una specie di algoritmo:
\begin{itemize}
	\item Per la prima riga si considera la sequenza di coefficienti ad indice pari
	\item Per la seconda riga i coefficienti di indice dispari
	\item Per la cella \((2,1)\) si esegue \((2,1)\cdot(1,2) - (1,1)\cdot(2,2)\).
		È possibile dividere o moltiplicare per la cella \((2,1)\) o
		sottomultipli, a patto che \emph{il segno non vari}.
	\item Per le celle successive sulla stessa riga, si trasla l'operazione
		mantenendo costante la prima colonna.
	\item Per le righe successive si riutilizzano i punti sopra citati.%
		\footnote{La cella di \(s^0\) ha quasi sempre lo stesso valore
			dell'ultima cella presente nella riga di \(s^2\)}
\end{itemize}
\[\begin{array}{r|rrrr}
	\tikzmark{g22m6}s^6 &   1 &   2 &  0 & -2	\\
	\tikzmark{g22m5}s^5 &   2 & - 1 & -2 	\\
	\tikzmark{g22m4}s^4 &   5 &   2 & -4 	\\
	\tikzmark{g22m3}s^3 & - 9 & - 2		\\
	\tikzmark{g22m2}s^2 &   8 & -36		\\
	\tikzmark{g22m1}s^1 & -85			\\
	\tikzmark{g22m0}s^0 & -36
\end{array}\]
\begin{tikzpicture}[overlay, remember picture, yshift=.25\baselineskip, shorten >=.5pt, shorten <=.5pt]
	\draw [->] ({pic cs:g22m6}) [bend right] to node[left]{\scriptsize p} ({pic cs:g22m5});
	\draw [->] ({pic cs:g22m5}) [bend right] to node[left]{\scriptsize p} ({pic cs:g22m4});
	\draw [->] ({pic cs:g22m4}) [bend right] to node[left]{\scriptsize v} ({pic cs:g22m3});
	\draw [->] ({pic cs:g22m3}) [bend right] to node[left]{\scriptsize v} ({pic cs:g22m2});
	\draw [->] ({pic cs:g22m2}) [bend right] to node[left]{\scriptsize v} ({pic cs:g22m1});
	\draw [->] ({pic cs:g22m1}) [bend right] to node[left]{\scriptsize p} ({pic cs:g22m0});
\end{tikzpicture}
\begin{itemize}
	\item Si segnano le permanenze (p) e le variazioni (v) di segno tra la prima cella
		di una riga e la cella della seguente.
	\item Le permanenze indicano il numero di poli con \(\Re s < 0\), mentre
		le variazioni i poli con \(\Re s > 0\).
\end{itemize}
In questo caso si hanno 3 permanenze e 3 variazioni: il sistema è \emph{instabile}.
\end{esempio}

\begin{esempio} Sia dato il seguente sistema:
\[
	G(s) = \frac{k}{s^3 - 12s +16}
\]
Il parametro \(k\) presente al numeratore non deve interessare perché si stanno
considerando solo i poli.

Si nota che manca il coefficiente di \(s^2\), che è \(0\).
Si procede con la tabella di Routh:
\[\begin{array}{r|rr}
	s^3 & 1 & -12 \\
	s^2 & 0 & 16
\end{array}\]
A questo punto è necessario fermarsi perché lo \(0\) non ha segno. Per ovviare ciò,
si segue un metodo di sostituzione della riga:
\begin{itemize}
	\item Si moltiplica la riga per \((-1)^h\) con \(h\) il numero di zeri
		incontrati fin'ora. (\(0;\; -16\))
	\item Si traslano a sinistra le celle (quelle in testa vanno in coda): (\(-16;\; 0\))
	\item Si effettua la somma in colonna tra la riga originale e quella
		ottenuta con questo metodo:
		\[\begin{array}{rr|r}
			  0 & 16 & +\\
			-16 &  0 & =\\
			\midrule
			-16 & 16
		\end{array}\]
	\item La riga ottenuta sostituisce quella originale.
\end{itemize}
\[\begin{array}{r|rr}
	\tikzmark{g23m3} s^3 &   1 & -12 \\
	\tikzmark{g23m2a}s^2 &   0 &  16 \\
	\midrule
	\tikzmark{g23m2b}s^2 & -16 &  16 \\
	\tikzmark{g23m1} s^1 & -11 	  \\
	\tikzmark{g23m0} s^0 &  16
\end{array}\]
\begin{tikzpicture}[overlay, remember picture, yshift=.25\baselineskip, shorten >=.5pt, shorten <=.5pt]
	\draw [->] ({pic cs:g23m3})  [bend right] to node[left]{\scriptsize v} ({pic cs:g23m2b});
	\draw [->] ({pic cs:g23m2b}) [bend right] to node[left]{\scriptsize p} ({pic cs:g23m1});
	\draw [->] ({pic cs:g23m1})  [bend right] to node[left]{\scriptsize v} ({pic cs:g23m0});
\end{tikzpicture}
Sono presenti due variazioni e una permanenza: il sistema è \emph{instabile}.
\end{esempio}

\begin{esempio} Si applichi il criterio di Routh al seguente sistema:
\[
	G(s) = \frac{s^5 + 4s + 1}{s^6 + 2s^5 + 8s^4 + 12s^3 + 20s^2 + 16s + 16}
\]
\[\begin{array}{r|rrrr}
	s^6 & 1 &  8 & 20 & 16	\\
	s^5 & 2 & 12 & 16 	\\
	s^4 & 1 &  6 &  8	\\
	s^3 & 0 &  0
\end{array}\]
A questo punto è necessario fermarsi perché si ha una riga completamente nulla.
Questo significa che esistono delle radici puramente immaginarie simmetriche
rispetto l'origine.

Si considerano i coefficienti della riga precedente e si costruisce un'equazione
di variabile \(s\): \(s^4 + 6s^2 + 8 = 0\). Derivandola si ha \(4s^3 + 12s = 0\).
La coppia di coefficienti di questa equazione costituiscono la nuova riga che
sostituisce quella nulla. Si nota che i coefficienti hanno multiplo in comune 4,
quindi dividento tutto per 4 si ha \((1;\; 3)\).
\[\begin{array}{r|rrrr}
	\tikzmark{g24m6} s^6 & 1 &  8 & 20 & 16	\\
	\tikzmark{g24m5} s^5 & 2 & 12 & 16 	\\
	\tikzmark{g24m4} s^4 & 1 &  6 &  8	\\
			 s^3 & 0 &  0		\\
	\midrule
	\tikzmark{g24m3} s^3 & 1 &  3		\\
	\tikzmark{g24m2} s^2 & 3 &  8		\\
	\tikzmark{g24m1} s^1 & 1			\\
	\tikzmark{g24m0} s^0 & 8
\end{array}\]
\begin{tikzpicture}[overlay, remember picture, yshift=.25\baselineskip, shorten >=.5pt, shorten <=.5pt]
	\draw [->] ({pic cs:g24m6}) [bend right] to node[left]{\scriptsize p} ({pic cs:g24m5});
	\draw [->] ({pic cs:g24m5}) [bend right] to node[left]{\scriptsize p} ({pic cs:g24m4});
	\draw [->] ({pic cs:g24m4}) [bend right] to node[left]{\scriptsize p} ({pic cs:g24m3});
	\draw [->] ({pic cs:g24m3}) [bend right] to node[left]{\scriptsize p} ({pic cs:g24m2});
	\draw [->] ({pic cs:g24m2}) [bend right] to node[left]{\scriptsize p} ({pic cs:g24m1});
	\draw [->] ({pic cs:g24m1}) [bend right] to node[left]{\scriptsize p} ({pic cs:g24m0});
\end{tikzpicture}
Nonostante siano tutte permanenze, bisogna tenere in considerazione che si è
sostituita una riga vuota e, come detto sopra, si hanno dei poli sull'asse
immaginario. Per verificarlo, basta considerare le soluzioni dell'equazione
precedente \(4s^3 + 12s = 0\) che ha soluzioni \(s_1 = 0\) e \(s_{2,3} = \pm\jmath\sqrt{3}\).
Per questo motivo il sistema è \emph{semplicemente stabile}.
\end{esempio}

\begin{esempio} Si applichi il criterio di Routh al seguente sistema:
\[
	G(s) = \frac{s^5 + 4s + 1}{s^5 + 5s^4 + 4s^3 + 120s^2 + 3s + 315}
\]
\[\begin{array}{r|rrr}
	\tikzmark{g25m5} s^5 & 1 & 4 & 3 \\
	\tikzmark{g25m4} s^4 & \cancelto{1}{3} & \cancelto{24}{120} & \cancelto{63}{315} \\
	\tikzmark{g25m3} s^3 & \cancelto{-1}{-20} & \cancelto{-3}{-60} \\
	\tikzmark{g25m2} s^2 & \cancelto{1}{21} & \cancelto{3}{63} \\
			 s^1 & 0 \\
	\midrule
	\tikzmark{g25m1} s^1 & 2 \\
	\tikzmark{g25m0} s^0 & 3
\end{array}\]
\begin{tikzpicture}[overlay, remember picture, yshift=.25\baselineskip, shorten >=.5pt, shorten <=.5pt]
	\draw [->] ({pic cs:g25m5}) [bend right] to node[left]{\scriptsize p} ({pic cs:g25m4});
	\draw [->] ({pic cs:g25m4}) [bend right] to node[left]{\scriptsize v} ({pic cs:g25m3});
	\draw [->] ({pic cs:g25m3}) [bend right] to node[left]{\scriptsize v} ({pic cs:g25m2});
	\draw [->] ({pic cs:g25m2}) [bend right] to node[left]{\scriptsize p} ({pic cs:g25m1});
	\draw [->] ({pic cs:g25m1}) [bend right] to node[left]{\scriptsize p} ({pic cs:g25m0});
\end{tikzpicture}
Il sistema presenta 3 permanenze e 2 variazioni, quindi è \emph{instabile}.
\end{esempio}

\section{Esercizi svolti}
\exercise{} Si applichi il criterio di Routh a
\[
	P(s) = s^4 + 4s^3 + 3s^2 + 8s + 5
\]
\[\begin{array}{r|rrr}
	\tikzmark{e21m4} s^4 & 1 & 3 & 5 \\
	\tikzmark{e21m3} s^3 & 4 & 8 \\
	\tikzmark{e21m2} s^2 & \cancelto{1}{4} & \cancelto{5}{20} \\
	\tikzmark{e21m1} s^1 & -12 \\
	\tikzmark{e21m0} s^0 & 5
\end{array}\]
\begin{tikzpicture}[overlay, remember picture, yshift=.25\baselineskip, shorten >=.5pt, shorten <=.5pt]
	\draw [->] ({pic cs:e21m4}) [bend right] to node[left]{\scriptsize p} ({pic cs:e21m3});
	\draw [->] ({pic cs:e21m3}) [bend right] to node[left]{\scriptsize p} ({pic cs:e21m2});
	\draw [->] ({pic cs:e21m2}) [bend right] to node[left]{\scriptsize v} ({pic cs:e21m1});
	\draw [->] ({pic cs:e21m1}) [bend right] to node[left]{\scriptsize v} ({pic cs:e21m0});
\end{tikzpicture}
Il sistema è \emph{instabile} con 2 poli aventi \(\Re s > 0\).


\exercise{} Si applichi il criterio di Routh a
\[
	P(s) = 8s^4 + 3s^3 + 7s^2 + 2s + 1
\]
\[\begin{array}{r|rrr}
	\tikzmark{e22m4} s^4 & 8 & 7 & 1 \\
	\tikzmark{e22m3} s^3 & 3 & 2 	 \\
	\tikzmark{e22m2} s^2 & 5 & 3	 \\
	\tikzmark{e22m1} s^1 & 1	 \\
	\tikzmark{e22m0} s^0 & 3
\end{array}\]
\begin{tikzpicture}[overlay, remember picture, yshift=.25\baselineskip, shorten >=.5pt, shorten <=.5pt]
	\draw [->] ({pic cs:e22m4}) [bend right] to node[left]{\scriptsize p} ({pic cs:e22m3});
	\draw [->] ({pic cs:e22m3}) [bend right] to node[left]{\scriptsize p} ({pic cs:e22m2});
	\draw [->] ({pic cs:e22m2}) [bend right] to node[left]{\scriptsize p} ({pic cs:e22m1});
	\draw [->] ({pic cs:e22m1}) [bend right] to node[left]{\scriptsize p} ({pic cs:e22m0});
\end{tikzpicture}
Il sistema presenta solo poli con \(\Re s < 0\), quindi è \emph{asintoticamente stabile}.


\exercise{} Si applichi il criterio di Routh a
\[
	P(s) = s^4 + s^3 + 5s^2 + 5s + 2
\]
\[\begin{array}{r|rrr}
	\tikzmark{e23m4} s^4 &  1 & 5 & 2 \\
	\tikzmark{e23m3} s^3 &  1 & 5	 \\
	s^2 &  0 & 2	 		 \\
	\midrule
	\tikzmark{e23m2} s^2 & -2 & 2	 \\
	\tikzmark{e23m1} s^1 & \cancelto{6}{12} \\
	\tikzmark{e23m0} s^0 & 2
\end{array}\]
\begin{tikzpicture}[overlay, remember picture, yshift=.25\baselineskip, shorten >=.5pt, shorten <=.5pt]
	\draw [->] ({pic cs:e23m4}) [bend right] to node[left]{\scriptsize p} ({pic cs:e23m3});
	\draw [->] ({pic cs:e23m3}) [bend right] to node[left]{\scriptsize v} ({pic cs:e23m2});
	\draw [->] ({pic cs:e23m2}) [bend right] to node[left]{\scriptsize v} ({pic cs:e23m1});
	\draw [->] ({pic cs:e23m1}) [bend right] to node[left]{\scriptsize p} ({pic cs:e23m0});
\end{tikzpicture}
Il sistema presenta 2 poli con \(\Re s > 0\), quindi è \emph{instabile}.


\exercise{} Si applichi il criterio di Routh a
\[
	P(s) = s^5 + s^4 + 4s^3 + 4s^2 + 7s + 7
\]
\[\begin{array}{r|rrr}
	\tikzmark{e24m5} s^5 & 1 & 4 & 7 \\
	\tikzmark{e24m4} s^4 & 1 & 4 & 7 \\
			 s^3 & 0 & 0	 \\
	\midrule
	\tikzmark{e24m3} s^3 & \cancelto{1}{4} & \cancelto{2}{8} \\
	\tikzmark{e24m2} s^2 & 2 & 7	\\
	\tikzmark{e24m1} s^1 & -3	\\
	\tikzmark{e24m0} s^0 & 7
\end{array}\]
\begin{tikzpicture}[overlay, remember picture, yshift=.25\baselineskip, shorten >=.5pt, shorten <=.5pt]
	\draw [->] ({pic cs:e24m5}) [bend right] to node[left]{\scriptsize p} ({pic cs:e24m4});
	\draw [->] ({pic cs:e24m4}) [bend right] to node[left]{\scriptsize p} ({pic cs:e24m3});
	\draw [->] ({pic cs:e24m3}) [bend right] to node[left]{\scriptsize p} ({pic cs:e24m2});
	\draw [->] ({pic cs:e24m2}) [bend right] to node[left]{\scriptsize v} ({pic cs:e24m1});
	\draw [->] ({pic cs:e24m1}) [bend right] to node[left]{\scriptsize v} ({pic cs:e24m0});
\end{tikzpicture}
Il sistema presenta 2 poli con \(\Re s > 0\), quindi è \emph{instabile}.


\exercise{} Si applichi il criterio di Routh a
\[
	P(s) = s^5 + 2s^4 + 4s^3 + 4s^2 + 3s + 2
\]
\[\begin{array}{r|rrr}
	\tikzmark{e25m5} s^5 & 1 & 4 & 3 \\
	\tikzmark{e25m4} s^4 & 2 & 4 & 2 \\
	\tikzmark{e25m3} s^3 & \cancelto{1}{4} & \cancelto{1}{4} \\
	\tikzmark{e25m2} s^2 & \cancelto{1}{2} & \cancelto{1}{2} \\
			 s^1 & 0 	 \\
	\midrule
	\tikzmark{e25m1} s^1 & 2 	 \\
	\tikzmark{e25m0} s^0 & 1
\end{array}\]
\begin{tikzpicture}[overlay, remember picture, yshift=.25\baselineskip, shorten >=.5pt, shorten <=.5pt]
	\draw [->] ({pic cs:e25m5}) [bend right] to node[left]{\scriptsize p} ({pic cs:e25m4});
	\draw [->] ({pic cs:e25m4}) [bend right] to node[left]{\scriptsize p} ({pic cs:e25m3});
	\draw [->] ({pic cs:e25m3}) [bend right] to node[left]{\scriptsize p} ({pic cs:e25m2});
	\draw [->] ({pic cs:e25m2}) [bend right] to node[left]{\scriptsize p} ({pic cs:e25m1});
	\draw [->] ({pic cs:e25m1}) [bend right] to node[left]{\scriptsize p} ({pic cs:e25m0});
\end{tikzpicture}
Il sistema presenta, al limite della stabilità, due poli puramente immaginari dati
dall'equazione \(s^2 + 1 = 0\), questa utilizzata per sostituire la riga nulla.
Quindi il sistema è \emph{semplicemente stabile}.


\exercise{} Si applichi il criterio di Routh e si stabiliscano i poli per
\[
	P(s) = s^6 + s^5 + 3s^4 + 3s^3 + 3s^2 + 2s + 1
\]
\[\begin{array}{r|rrrr}
	\tikzmark{e26m6} s^6 &  1 & 3 & 3 & 1	\\
	\tikzmark{e26m5} s^5 &  1 & 3 & 2	\\
			 s^4 &  0 & 1 & 1	\\
	\midrule
	\tikzmark{e26m4} s^4 & -1 & 0 & -1	\\
	\tikzmark{e26m3} s^3 & \cancelto{1}{3} & \cancelto{1}{3} \\
	\tikzmark{e26m2} s^2 &  1 & 1		\\
			 s^1 &  0		\\
	\midrule
	\tikzmark{e26m1} s^1 &  2		\\
	\tikzmark{e26m0} s^0 &  1
\end{array}\]
\begin{tikzpicture}[overlay, remember picture, yshift=.25\baselineskip, shorten >=.5pt, shorten <=.5pt]
	\draw [->] ({pic cs:e26m6}) [bend right] to node[left]{\scriptsize p} ({pic cs:e26m5});
	\draw [->] ({pic cs:e26m5}) [bend right] to node[left]{\scriptsize v} ({pic cs:e26m4});
	\draw [->] ({pic cs:e26m4}) [bend right] to node[left]{\scriptsize v} ({pic cs:e26m3});
	\draw [->] ({pic cs:e26m3}) [bend right] to node[left]{\scriptsize p} ({pic cs:e26m2});
	\draw [->] ({pic cs:e26m2}) [bend right] to node[left]{\scriptsize p} ({pic cs:e26m1});
	\draw [->] ({pic cs:e26m1}) [bend right] to node[left]{\scriptsize p} ({pic cs:e26m0});
\end{tikzpicture}
Il sistema deve avere 6 poli (equazione di grado 6): infatti presenta 2 poli con
\(\Re s_i > 0\), 2 poli puramente immaginari \(s_{1,2} = \pm\jmath\) soluzioni
dell'equazione \(s^2 + 1 = 0\), ricavata per sostituire la riga nulla \(s^1\),
e 2 poli con \(\Re s_i < 0\). Per concludere, il sistema è \emph{instabile}.


\exercise{Parametrico}
Si applichi il criterio di Routh a
\[
	P(s) = s^3 + 5s^2 + 6s + k
\]
\[\begin{array}{r|rr}
	\tikzmark{e27m3} s^3 & 1 & 6 \\
	\tikzmark{e27m2} s^2 & 5 & k \\
	\tikzmark{e27m1} s^1 & 30-k  \\
	\tikzmark{e27m0} s^0 & k
\end{array}\]
\begin{tikzpicture}[overlay, remember picture, yshift=.25\baselineskip, shorten >=.5pt, shorten <=.5pt]
	\draw [->] ({pic cs:e27m3}) [bend right] to node[left]{\scriptsize p} ({pic cs:e27m2});
	\draw [->] ({pic cs:e27m2}) [bend right] to node[left]{\scriptsize ?} ({pic cs:e27m1});
	\draw [->] ({pic cs:e27m1}) [bend right] to node[left]{\scriptsize ?} ({pic cs:e27m0});
\end{tikzpicture}

\begin{itemize}
	\item Se \(k > 30\):
		\[
			\begin{cases}
				2 \text{ poli con } \Re s_i > 0 \\
				1 \text{ polo con } \Re s   < 0
			\end{cases} \implies \text{sistema \emph{instabile}}
		\]
	\item Se \(k = 30\):
		\[\begin{array}{r|rr}
			\tikzmark{e27bm3} s^3 & 1 & 6	\\
			\tikzmark{e27bm2} s^2 & 5 & 30	\\
					  s^1 & 0 	\\
			\midrule
			\tikzmark{e27bm1} s^1 & 10	\\
			\tikzmark{e27bm0} s^0 & 30
		\end{array}\]
		\begin{tikzpicture}[overlay, remember picture, yshift=.25\baselineskip, shorten >=.5pt, shorten <=.5pt]
			\draw [->] ({pic cs:e27bm3}) [bend right] to node[left]{\scriptsize p} ({pic cs:e27bm2});
			\draw [->] ({pic cs:e27bm2}) [bend right] to node[left]{\scriptsize p} ({pic cs:e27bm1});
			\draw [->] ({pic cs:e27bm1}) [bend right] to node[left]{\scriptsize p} ({pic cs:e27bm0});
		\end{tikzpicture}
		Le soluzioni dell'equazione \(5s^2 + 30 = 0\) sono puramente
		immaginari, quindi il sistema è \emph{semplicemente stabile}
	\item Se \(0 < k < 30\): 3 poli con \(\Re s_i < 0 \implies\) sistema \emph{asintoticamente stabile}
	\item Se \(k < 0\):
		\[
			\begin{cases}
				1 \text{ polo con } \Re s   > 0 \\
				2 \text{ poli con } \Re s_i < 0
			\end{cases} \implies \text{sistema \emph{instabile}}
		\]
	\item Se \(k = 0\):
		\[\begin{array}{r|rr}
			\tikzmark{e27cm3} s^3 & 1 & 6	\\
			\tikzmark{e27cm2} s^2 & 5 & 0	\\
			\tikzmark{e27cm1} s^1 & 30	\\
					  s^0 & 0	\\
			\midrule
			\tikzmark{e27cm0} s^0 & 30
		\end{array}\]
		\begin{tikzpicture}[overlay, remember picture, yshift=.25\baselineskip, shorten >=.5pt, shorten <=.5pt]
			\draw [->] ({pic cs:e27cm3}) [bend right] to node[left]{\scriptsize p} ({pic cs:e27cm2});
			\draw [->] ({pic cs:e27cm2}) [bend right] to node[left]{\scriptsize p} ({pic cs:e27cm1});
			\draw [->] ({pic cs:e27cm1}) [bend right] to node[left]{\scriptsize p} ({pic cs:e27cm0});
		\end{tikzpicture}
		La soluzione dell'equazione \(30s = 0\) coincide con l'origine,
		quindi il sistema è \emph{semplicemente stabile}
\end{itemize}


\exercise{Parametrico} Si applichi il criterio di Routh e si stabiliscano i poli per
\[
	G_0 (s) = \frac{s-1}{s^2 + 5s^2 + (k-6)s + k}
\]
\[\begin{array}{r|rr}
	\tikzmark{e28am3} s^3 & 	  1 & k-6 \\
	\tikzmark{e28am2} s^2 & 	  5 &   k \\
	\tikzmark{e28am1} s^1 & 2k-15 	  	  \\
	\tikzmark{e28am0} s^0 & k
\end{array}\]
\begin{tikzpicture}[overlay, remember picture, yshift=.25\baselineskip, shorten >=.5pt, shorten <=.5pt]
	\draw [->] ({pic cs:e28am3}) [bend right] to node[left]{\scriptsize p} ({pic cs:e28am2});
	\draw [->] ({pic cs:e28am2}) [bend right] to node[left]{\scriptsize ?} ({pic cs:e28am1});
	\draw [->] ({pic cs:e28am1}) [bend right] to node[left]{\scriptsize ?} ({pic cs:e28am0});
\end{tikzpicture}
\begin{itemize}
	\item Se \(k < 0\):
		\[\begin{cases}
			1 \text{ polo con } \Re s   > 0 \\
			2 \text{ poli con } \Re s_i < 0
		\end{cases} \implies \text{sistema \emph{instabile}}\]
	\item Se \(k = 0\):
		\[\begin{array}{r|rr}
			\tikzmark{e28bm3} s^3 &   1 & -6\\
			\tikzmark{e28bm2} s^2 &   5 & 0	\\
			\tikzmark{e28bm1} s^1 & -30	\\
					  s^0 &   0	\\
			\midrule
			\tikzmark{e28bm0} s^0 & -30
		\end{array}\]
		\begin{tikzpicture}[overlay, remember picture, yshift=.25\baselineskip, shorten >=.5pt, shorten <=.5pt]
			\draw [->] ({pic cs:e28bm3}) [bend right] to node[left]{\scriptsize p} ({pic cs:e28bm2});
			\draw [->] ({pic cs:e28bm2}) [bend right] to node[left]{\scriptsize v} ({pic cs:e28bm1});
			\draw [->] ({pic cs:e28bm1}) [bend right] to node[left]{\scriptsize p} ({pic cs:e28bm0});
		\end{tikzpicture}
		\[\begin{cases}
			1 \text{ polo con } \Re s > 0	\\
			1 \text{ polo all'origine} 	\\
			1 \text{ polo con } \Re s < 0
		\end{cases} \implies \text{sistema \emph{instabile}}\]
	\item Se \(0 < k < \frac{15}{2}\):
		\[\begin{cases}
			2 \text{ poli con } \Re s_i > 0 \\
			1 \text{ polo con } \Re s   < 0
		\end{cases} \implies \text{sistema \emph{instabile}}\]
	\item Se \(k = \frac{15}{2}\):
		\[\begin{array}{r|rr}
			\tikzmark{e28dm3} s^3 &  1 & \frac{15}{2} - 6	\\
			\tikzmark{e28dm2} s^2 &  5 & \frac{15}{2}	\\
					  s^1 &  0			\\
			\midrule
			\tikzmark{e28dm1} s^1 &  2			\\
			\tikzmark{e28dm0} s^0 & 15
		\end{array}\]
		\begin{tikzpicture}[overlay, remember picture, yshift=.25\baselineskip, shorten >=.5pt, shorten <=.5pt]
			\draw [->] ({pic cs:e28dm3}) [bend right] to node[left]{\scriptsize p} ({pic cs:e28dm2});
			\draw [->] ({pic cs:e28dm2}) [bend right] to node[left]{\scriptsize p} ({pic cs:e28dm1});
			\draw [->] ({pic cs:e28dm1}) [bend right] to node[left]{\scriptsize p} ({pic cs:e28dm0});
		\end{tikzpicture}
		\[\begin{cases}
			2 \text{ poli puramente immaginari} \\
			1 \text{ polo con } \Re s < 0
		\end{cases} \implies \text{sistema \emph{semplicemente stabile}}\]
	\item Se \(k > \frac{15}{2}\):
		\[\begin{cases}
			3 \text{ poli con } \Re s_i < 0
		\end{cases} \implies \text{sistema \emph{asintoticamente stabile}}\]
\end{itemize}


\exercise{Sistema a retroazione negativa} Dato il sistema
\begin{center}\begin{tikzpicture}[auto,node distance=2cm,>=latex']
	\node [input] (x) {};
	\node [sum, right of=x] (sum) {};
	\node [block, right of=sum, node distance=1.5cm] (G1) {\(G_c (s)\)};
	\node [block, right of=G1] (G2) {\(G_p (s)\)};
	\node [output, right of=G2] (y) {};
	\node [tmp, below of=G1, node distance=1cm] (tmp) {};
	\draw [->] (x) -- (sum);
	\draw [->] (sum) -- (G1);
	\draw [->] (G1) -- (G2);
	\draw [->] (G2) -- node[name=retro]{} (y);
	\draw [->] (retro) |- (tmp) -| node[pos=0.9,anchor=east]{\(-\)} (sum);
\end{tikzpicture}\end{center}
con
\begin{align*}
	G_c (s) &= k \frac{s+1}{s-2} \\
	G_p (s) &= \frac{1}{s^2 + 2s + 10}
\end{align*}
si applichi il criterio di Routh e si discutano i poli al variare del parametro \(k\).

\paragraph{Soluzione}
\(G_c\) e \(G_p\) sono in serie, quindi
\[
	G(s) = G_c(s)G_p(s) = \frac{k(s+1)}{(s-2)(s^2 + 2s + 10)}
\]
Si considera \((s-2)(s^2+2s+10)+k(s+1) = s^3 \cancel{-2s^2} \cancel{+2s^2} -4s +10s -20 +ks +k = s^3 +(k+6)s -20+k\),
quindi \[P(s) = s^3 +(k+6)s -20+k\]
\[\begin{array}{r|rr}
	s^3 & 1 & k+6 \\
	s^2 & 0 & -20+k \\
	\midrule
	s^2 & 20-k & -20+k \\
	s^1 & -k^2+13k+140
\end{array}\]
È possibile semplificare l'equazione \(-k^2 +13k +140 = 0\) trovando le sue radici:
\begin{align*}
	k^2 -13k -140 &= 0 \\
	k_{1,2} = \frac{13 \pm27}{2} &= \begin{cases} -7 \\ 20 \end{cases} \\
	\implies k^2 -13k -140 &= (k +7) (k -20) \\
	-k^2 +13k +140 &= (k +7) (20 -k) \\
	\implies \frac{-k^2 +13k +140}{20 -k} &= k+7
\end{align*}
Di conseguenza
\[\begin{array}{r|rr}
	s^3 &    1 &   k+6 \\
	s^2 & 20-k & -20+k \\
	s^1 &  k+7 	   \\
	s^0 & -20+k
\end{array}\]
\begin{itemize}
	\item Se \(k < -7\):
		\[\begin{cases}
			1 \text{ polo con } \Re s > 0
		\end{cases}\]
	\item Se \(k = -7\):
		\[\begin{array}{r|rr}
			\tikzmark{e29am3} s^3 &  1 & -1 \\
			\tikzmark{e29am2} s^2 &  1 & -1 \\
					  s^1 &  0	\\
			\midrule
			\tikzmark{e29am1} s^1 &  2	\\
			\tikzmark{e29am0} s^0 & -1
		\end{array}\]
		\begin{tikzpicture}[overlay, remember picture, yshift=.25\baselineskip, shorten >=.5pt, shorten <=.5pt]
			\draw [->] ({pic cs:e29am3}) [bend right] to node[left]{\scriptsize p} ({pic cs:e29am2});
			\draw [->] ({pic cs:e29am2}) [bend right] to node[left]{\scriptsize p} ({pic cs:e29am1});
			\draw [->] ({pic cs:e29am1}) [bend right] to node[left]{\scriptsize v} ({pic cs:e29am0});
		\end{tikzpicture}
		\[\begin{cases}
			2 \text{ poli con } \Re s > 0	\\
		\end{cases} \implies \text{sistema \emph{instabile}}\]
	\item Se \(-7 < k < 20\):
		\[\begin{cases}
			1 \text{ polo con } \Re s   > 0
		\end{cases} \implies \text{sistema \emph{instabile}}\]
	\item Se \(k = 20\):
		\[\begin{array}{r|rr}
			\tikzmark{e29bm3} s^3 &  1 & 26 \\
					  s^2 &  0 &  0 \\
			\midrule
			\tikzmark{e29bm2} s^2 &  3 & 26 \\
			\tikzmark{e29bm1} s^1 & 52	\\
			\tikzmark{e29bm0} s^0 & 26
		\end{array}\]
		\begin{tikzpicture}[overlay, remember picture, yshift=.25\baselineskip, shorten >=.5pt, shorten <=.5pt]
			\draw [->] ({pic cs:e29bm3}) [bend right] to node[left]{\scriptsize p} ({pic cs:e29bm2});
			\draw [->] ({pic cs:e29bm2}) [bend right] to node[left]{\scriptsize p} ({pic cs:e29bm1});
			\draw [->] ({pic cs:e29bm1}) [bend right] to node[left]{\scriptsize p} ({pic cs:e29bm0});
		\end{tikzpicture}
		\[\begin{cases}
			1 \text{ polo all'origine} 		\\
			2 \text{ poli puramente immaginari}
		\end{cases} \implies \text{sistema \emph{semplicemente stabile}}\]
	\item Se \(k > 20\):
		\[\begin{cases}
			2 \text{ poli con } \Re s_i > 0
		\end{cases}\]
	\item Il sistema non è asintoticamente stabile \(\forall k\)
\end{itemize}

\exercise{Parametrico} Si applichi il criterio di Routh e si discutino i poli al variare di \(k\) per
\[
	G(s) = \frac{250k (s-3)}{(s+3)(s^2 +50s +2500)}
\]
\begin{align*}
	P(s) &= 250k (s-3) + (s+3)(s^2 +50s +2500) = 		   \\
	     &= 250ks -705k +s^3 +3s^2 +50s^2 +150s +2500s +7500 = \\
	     &= s^3 +53s^2 +(250k +2650)s +7500 -750k
\end{align*}
\[\begin{array}{r|rr}
	\tikzmark{e210am3} s^3 & 1 & 250k +2650 \\
	\tikzmark{e210am2} s^2 & 53 & 7500 -750k \\
	\tikzmark{e210am1} s^1 & 280k +2659 \\
	\tikzmark{e210am0} s^0 & 10-k
\end{array}\]
\begin{tikzpicture}[overlay, remember picture, yshift=.25\baselineskip, shorten >=.5pt, shorten <=.5pt]
	\draw [->] ({pic cs:e210am3}) [bend right] to node[left]{\scriptsize p} ({pic cs:e210am2});
	\draw [->] ({pic cs:e210am2}) [bend right] to node[left]{\scriptsize ?} ({pic cs:e210am1});
	\draw [->] ({pic cs:e210am1}) [bend right] to node[left]{\scriptsize ?} ({pic cs:e210am0});
\end{tikzpicture}
\begin{itemize}
	\item Se \(k < -\frac{2659}{280}\):
		\[\begin{cases}
			2 \text{ poli con } \Re s_i > 0 \\
			1 \text{ polo con } \Re s   < 0
		\end{cases} \implies \text{sistema \emph{instabile}}\]
	\item Se \(k = -\frac{2659}{280}\):
		\[\begin{array}{r|rr}
			\tikzmark{e210bm3} s^3 & 1 & 250k + 2650 \\
			\tikzmark{e210bm2} s^2 & 53 & 7500 -750k \\
					   s^1 & 0 \\
			\midrule
			\tikzmark{e210bm1} s^1 & 106 \\
			\tikzmark{e210bm0} s^0 & 7500 -750k
		\end{array}\]
		\begin{tikzpicture}[overlay, remember picture, yshift=.25\baselineskip, shorten >=.5pt, shorten <=.5pt]
			\draw [->] ({pic cs:e210bm3}) [bend right] to node[left]{\scriptsize p} ({pic cs:e210bm2});
			\draw [->] ({pic cs:e210bm2}) [bend right] to node[left]{\scriptsize p} ({pic cs:e210bm1});
			\draw [->] ({pic cs:e210bm1}) [bend right] to node[left]{\scriptsize p} ({pic cs:e210bm0});
		\end{tikzpicture}
		\[\begin{cases}
			2 \text{ poli puramente immaginari} \\
			1 \text{ polo con } \Re s < 0
		\end{cases} \implies \text{sistema \emph{semplicemente stabile}}\]
	\item Se \(-\frac{2659}{280} < k < 10\):
		\[\begin{cases}
			3 \text{ poli con } \Re s_i < 0
		\end{cases} \implies \text{sistema \emph{asintoticamente stabile}}\]
	\item Se \(k = 10\):
		\[\begin{array}{r|rr}
			\tikzmark{e210cm3} s^3 & 1 & 5150 \\
			\tikzmark{e210cm2} s^2 & 53 & 0 \\
			\tikzmark{e210cm1} s^1 & 5459 \\
					   s^0 & 0 \\
			\midrule
			\tikzmark{e210cm0} s^0 & 5459
		\end{array}\]
		\begin{tikzpicture}[overlay, remember picture, yshift=.25\baselineskip, shorten >=.5pt, shorten <=.5pt]
			\draw [->] ({pic cs:e210cm3}) [bend right] to node[left]{\scriptsize p} ({pic cs:e210cm2});
			\draw [->] ({pic cs:e210cm2}) [bend right] to node[left]{\scriptsize p} ({pic cs:e210cm1});
			\draw [->] ({pic cs:e210cm1}) [bend right] to node[left]{\scriptsize p} ({pic cs:e210cm0});
		\end{tikzpicture}
		\[\begin{cases}
			1 \text{ polo all'origine} \\
			2 \text{ poli con } \Re s_i < 0
		\end{cases} \implies \text{sistema \emph{semplicemente stabile}}\]
	\item Se \(k > 10\):
		\[\begin{cases}
			1 \text{ polo con } \Re s   > 0 \\
			2 \text{ poli con } \Re s_i < 0
		\end{cases} \implies \text{sistema \emph{instabile}}\]
\end{itemize}


\exercise{Parametrico} Si applichi il criterio di Routh e si discutino i
poli al variare di \(k\) al seguente sistema
\[
	G_c(s) = k \qquad G_p(s) = \frac{s-1}{s (s+1) (s^2 +8s +25)}
\]

La funzione di trasferimento e \(P(s)\) sono dati da
\begin{align*}
	G(s) &= G_c(s) G_p(s) = \frac{k(s-1)}{s (s+1) (s^2 +8s +25)} \\
	P(s) &= k(s-1) + s (s+1) (s^2 +8s +25) = ks -k +(s^2 +s)(s^2 +8s +25) = \\
	     &= ks -k +s^4 +s^3 +8s^3 +8s^2 +25s^2 +25s = \\
	     &= s^4 +9s^3 +33s^2 +(25+k)s -k
\end{align*}
\[\begin{array}{r|rrr}
	s^4 & 1 & 33 & -k \\
	s^3 & 9 & 25+k \\
	s^2 & 272-k & -9k
\end{array}\]
Per ricavare la riga successiva:
\begin{align*}
	(272-k)(25+k) + 81k &= -k^2 +328k +6800 \\
	\implies k^2 -328k -6800 &= 0\\
	k_{1,2} = 164 \pm 36\sqrt{26} &\approx \begin{cases} 347.6 \\ -19.6 \end{cases}
\end{align*}
quindi
\[\begin{array}{r|rrr}
	\tikzmark{e211am4} s^4 & 1 & 33 & -k 	\\
	\tikzmark{e211am3} s^3 & 9 & 25+k 		\\
	\tikzmark{e211am2} s^2 & 272-k & -9k	\\
	\tikzmark{e211am1} s^1 & -k^2 +328k +6800	\\
	\tikzmark{e211am0} s^0 & -9k
\end{array}\]
\begin{tikzpicture}[overlay, remember picture, yshift=.25\baselineskip, shorten >=.5pt, shorten <=.5pt]
	\draw [->] ({pic cs:e211am4}) [bend right] to node[left]{\scriptsize p} ({pic cs:e211am3});
	\draw [->] ({pic cs:e211am3}) [bend right] to node[left]{\scriptsize ?} ({pic cs:e211am2});
	\draw [->] ({pic cs:e211am2}) [bend right] to node[left]{\scriptsize ?} ({pic cs:e211am1});
	\draw [->] ({pic cs:e211am1}) [bend right] to node[left]{\scriptsize ?} ({pic cs:e211am0});
\end{tikzpicture}
I casi sono
\[\begin{cases}
	272-k > 0 \rightarrow k < 272 \\
	-9k > 0 \rightarrow k < 0 \\
	-k^2 +328k +6800 > 0 \rightarrow 164 -36\sqrt{26} < k < 164 +36\sqrt{26}
\end{cases}\]

\begin{itemize}
	\item Se \(164-36\sqrt{26} < k < 0\): sistema \emph{asintoticamente stabile}
	\item Se \(k = 0\): 1 polo nell'origine \(\implies\) sistema \emph{semplicemente stabile}
	\item Se \(k = 164-36\sqrt{26}\): 2 poli puramente immaginari \(\implies\) sistema \emph{semplicemente stabile}
	\item Se \(0 < k < 272\): 1 polo con \(\Re s > 0 \implies\) sistema \emph{instabile}
	\item Se \(k = 272\): 1 polo nell'origine, 2 poli puramente immaginari e 1 polo con \(\Re s > 0\): sistema \emph{instabile}
	\item Se \(272 < k < 164+36\sqrt{26}\): 3 poli con \(\Re s_i > 0 \implies\) sistema \emph{instabile}
	\item Se \(k = 164+36\sqrt{26}\): 1 polo con \(\Re s > 0 \implies\) sistema \emph{instabile}
	\item Se \(k > 164+36\sqrt{26}\): 1 polo con \(\Re s > 0 \implies\) sistema \emph{instabile}
\end{itemize}


\exercise{} Si applichi il criterio di Routh a
\[
	G_0(s) = \frac{s-2}{s^5 +5s^4 +11s^3 +23s^2 +28s +12}
\]
\[\begin{array}{r|rrr}
	\tikzmark{e212m5} s^5 & 1 & 11 & 28 \\
	\tikzmark{e212m4} s^4 & 5 & 23 & 12 \\
	\tikzmark{e212m3} s^3 & \cancelto{1}{32} & \cancelto{4}{128} \\
	\tikzmark{e212m2} s^2 & 3 & 12	   \\
			 s^1 & 0	   \\
	\midrule
	\tikzmark{e212m1} s^1 & 6	   \\
	\tikzmark{e212m0} s^0 & 12
\end{array}\]
\begin{tikzpicture}[overlay, remember picture, yshift=.25\baselineskip, shorten >=.5pt, shorten <=.5pt]
	\draw [->] ({pic cs:e212m5}) [bend right] to node[left]{\scriptsize p} ({pic cs:e212m4});
	\draw [->] ({pic cs:e212m4}) [bend right] to node[left]{\scriptsize p} ({pic cs:e212m3});
	\draw [->] ({pic cs:e212m3}) [bend right] to node[left]{\scriptsize p} ({pic cs:e212m2});
	\draw [->] ({pic cs:e212m2}) [bend right] to node[left]{\scriptsize p} ({pic cs:e212m1});
	\draw [->] ({pic cs:e212m1}) [bend right] to node[left]{\scriptsize p} ({pic cs:e212m0});
\end{tikzpicture}
\[\begin{cases}
	2 \text{ poli puramente immaginari} \\
	3 \text{ poli con } \Re s_i < 0
\end{cases} \implies \text{sistema \emph{semplicemente stabile}}\]



\end{document}
